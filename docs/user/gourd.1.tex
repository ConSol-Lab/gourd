\documentclass[a4paper,english]{article}
\usepackage{a4wide}
\usepackage{babel}
\usepackage{verbatim}

\usepackage{changepage}

\usepackage[bookmarksopen,bookmarksnumbered]{hyperref}

\usepackage[fancyhdr]{latex2man}

\usepackage{xspace}



\newcommand{\thecmd}{gourd}
\newcommand{\thecommand}{GOURD}
\newcommand{\mansection}{1}
\newcommand{\mansectionname}{DelftBlue Tools Manual}
\newcommand{\mandate}{25 MARCH 2025}
\setDate{25 MARCH 2025}
\setVersionWord{Version:}
\setVersion{1.2.0}


\fancyhead[L, R]{\textit{\thecommand}(\mansection)}
\fancyhead[C]{\textsc{\mansectionname}}
\fancyfoot[C]{\mandate}
\fancyfoot[R]{\thepage}
\renewcommand{\headrulewidth}{0pt}

\renewcommand{\Prog}[1]{\textbf{#1}}                      % Program name

\renewenvironment{abstract}{\noindent\bfseries NAME\normalfont\vspace{0.2cm}}{}

\renewenvironment{Name}[5]{
\title{#5}
\author{#3}
\date{\@LM@Date\\{\small Version \@LM@Version}}
\begin{abstract}
}{
\end{abstract}
\vspace{-0.8cm}
}

\renewenvironment{Description}[1][]{
\begin{list}{}{
 \ifthenelse{\equal{#1}{}}{
   % optional argument not given
     \labelwidth\z@ \itemindent-\leftmargin
     \let\makelabel\descriptionlabel
     \renewcommand{\makelabel}[1]{\hspace\labelsep\normalfont\bfseries##1}
  }{
   % optional argument given
   \settowidth{\labelwidth}{\normalfont\bfseries#1}
   \setlength{\leftmargin}{\labelwidth}
   \addtolength{\leftmargin}{\labelsep}
   \renewcommand{\makelabel}[1]{\normalfont\bfseries##1\hfil}
  }}
}{
\end{list}
}

\usepackage{titlesec}
\titleformat{\section}
  {\normalfont\normalsize\bfseries}{\thesection}{0.5em}{}

\titleformat{\subsection}
  {\normalfont\normalsize\bfseries}{\thesubsection}{0.5em}{}

\setlength{\parindent}{0pt}
\setlength{\parskip}{6pt}

% Improve the \section command if in a LaTeX environment.
%@% IF LATEX %@%
\let\oldsection\section
\renewcommand{\section}[1]{%
\end{adjustwidth}%
\oldsection*{#1}%
\begin{adjustwidth}{18pt}{0pt}%
}
\let\oldsubsection\subsection
\renewcommand{\subsection}[1]{%
\oldsubsection*{#1}%
}
\let\oldsubsubsection\subsubsection
\renewcommand{\subsubsection}[1]{%
\oldsubsubsection*{#1}%
}

\newcommand{\ddash}{-{}-}
%@% END-IF %@%


\usepackage{mathspec}
\setmainfont[Mapping=tex-text, FakeBold=1]{Linux Libertine O}
\setmathfont(Digits,Greek,Latin)[Numbers=OldStyle, FakeBold=1]{Linux Libertine O}

\begin{document}
    \pagestyle{fancy}


    \begin{Name}{1}{gourd}{gourd}{DelftBlue Tools Manual}{Gourd}
%@% IF LATEX %@%
\begin{adjustwidth}{18pt}{0pt}
%@% END-IF %@%

        \Prog{gourd} - A tool for scheduling parallel runs for algorithm comparisons.

%@% IF LATEX %@%
\end{adjustwidth}
%@% END-IF %@%
    \end{Name}


%@% IF LATEX %@%
\begin{adjustwidth}{18pt}{0pt}
%@% END-IF %@%
    \section{SYNOPSIS}

        \Prog{gourd} \Arg{command}
        \oOpt{-s}
        \oOptArg{-c}{ filename}
        \oOpt{-v|-vv}
        \oOpt{-d}
        \oOpt{-h}

    \section{DESCRIPTION}

        \Prog{gourd} is a tool that schedules parallel runs for algorithm comparisons.
        Given the parameters of the experiment, a number of test datasets, and algorithm implementations to compare,
        \Prog{gourd} runs the experiment in parallel and provides many options for processing its results.
        While originally envisioned for the DelftBlue supercomputer at Delft University of Technology,
        \Prog{gourd} can replicate the experiment on any cluster computer with the \Prog{Slurm} scheduler,
        on any UNIX-like system, and on Microsoft Windows.

        New to \Prog{gourd}? Go directly to the \Prog{gourd-tutorial(7)} manual.


    \section{GLOBAL OPTIONS}

        The following options apply to all \Prog{gourd} commands.
        These will be referred to as \emph{GLOBAL OPTIONS} throughout the manual.

        \begin{Description}[Options]
            \item[\OptArg{-c}{ filename}, \OptArg{\ddash config}{ filename}]
            Tell \Prog{gourd} to use the given filename as \File{gourd.toml}, the configuration
            file that defines the experimental setup.
            By default, the file is expected in the current working directory at \File{./gourd.toml}.
            \item[\Opt{-d}, \Opt{\ddash dry-run}]
            Run \Prog{gourd} in dry-run mode, printing all operations (such as writing to files or scheduling runs)
            without executing them.
            \item[\Opt{-h}, \Opt{\ddash help}]
            Display usage instructions for the \Prog{gourd} utility or any of its commands.
            This option extends to all the subcommands of \Prog{gourd}: for example, running
            \Prog{gourd} \Arg{status} \Arg{-h} will display help about the \Arg{status} subcommand.
            \item[\Opt{-s}, \Opt{\ddash script}]
            Tell \Prog{gourd} to use a script-friendly interface, that is, one that does not use
            interactive user prompts.
            \item[\Opt{-v}, \Opt{-vv}, \Opt{\ddash verbose}]
            Run \Prog{gourd} with verbose debugging output, where \Opt{-vv} enables even more logging.
        \end{Description}

    \section{COMMANDS}

        Using \Prog{gourd} is as simple as invoking one of its commands, such as
        \Prog{gourd}~\Arg{init}.
        Command-line arguments are generally not necessary; to design and run
        experiments, a \File{gourd.toml} file should be in the current directory.
        The following is a summary of available commands.

        \begin{Description}[Commands]
            \item[\Prog{gourd} \Arg{run}]
              Create an experiment from configuration and run it on \Prog{Slurm} or the local machine.
            \item[\Prog{gourd} \Arg{init}]
              Set up a template of an experiment configuration.
            \item[\Prog{gourd} \Arg{status}]
              Display the status of a running or completed experiment.
            \item[\Prog{gourd} \Arg{continue}]
              Schedule the incomplete part of a partial experiment.
            \item[\Prog{gourd} \Arg{cancel}]
              Cancel scheduled runs.
            \item[\Prog{gourd} \Arg{rerun}]
              Rerun (possibly) failed runs.
            \item[\Prog{gourd} \Arg{analyse}]
              Output metrics of completed runs.
            \item[\Prog{gourd} \Arg{set-limits}]
              Change \Prog{Slurm} resource limits for runs not yet scheduled.
            \item[\Prog{gourd} \Arg{version}]
              Show the software version.
        \end{Description}

        \subsection{GOURD RUN}

            \subsubsection{Summary}
                The \Prog{gourd} \Arg{run} command uses the provided configuration and runs an
                experiment.
                Using either \Arg{local} or \Arg{slurm}, it is possible for the execution
                to run on the local machine, or be scheduled using Slurm on a cluster computer.
                Using Slurm, additional configuration arguments are required; see
                \Prog{gourd.toml(5)}.

                In principle, however, \Prog{gourd} \Arg{run} is used in the same way for both
                Slurm and local execution.
                A \File{gourd.toml} configuration file should be present in the current directory,
                formally describing the experiment that is to be created and run.
                Because most options are specified in this file, it is usually sufficient to type
                \Prog{gourd} \Arg{run} \Arg{slurm}|\Arg{local} to run an experiment.

                See the manual page \Prog{gourd-tutorial(7)} for a step-by-step guide on
                designing experiments to run.


            \subsubsection{Synopsis}
                \Prog{gourd}
                \Arg{run}
                \Arg{slurm}|\Arg{local}
                \oOpt{GLOBAL OPTIONS}
                \oOpt{\ddash force}
                \oOpt{\ddash sequential}

            \subsubsection{Subcommands}
                \begin{Description}[Subcommands]
                    \item[\Arg{local}]
                        Runs the experiment locally, such that all programs are executed directly on the computer
                        that \Prog{gourd} is run from.

                        This is useful for running \textbf{small parts} of an experiment on
                        a personal computer or a login node, allowing you to test that programs
                        are being called correctly and that the configuration is valid.
                        Please note that \emph{local} is \textbf{NOT} intended for running full
                        experiments on a \Prog{Slurm}-equipped cluster.
                        \emph{Local} will use the login node only and not the actual
                        supercomputer.

                        Running using \Arg{run} \Arg{local} will perform the experiments in
                        parallel based on the number of available CPU cores.
                        Resource limits set in the configuration will not be honoured.
                        While running, experiment status is displayed continuously (see the
                        \Prog{gourd} \Arg{status} command) until all runs have finished executing.
                        Typing Control+C into the terminal will stop the runs.

                        \Arg{local} can additionally take more options:
                            \begin{Description}[Options]
                                \item[\Opt{\ddash force}]
                                  \Prog{gourd} will, by default, refuse to run large experiments on \Arg{local}. This is
                                  because doing so may rapidly use up too many file descriptors on some operating systems.
                                  To ignore the warning and run the experiment anyways, use this optional flag.
                                  If the resources are exhausted too quickly, an error will be displayed.

                                \item[\Opt{\ddash sequential}]
                                  By default, runs execute concurrently with a level of parallelism.
                                  This option can be specified to force the runs to run sequentially, that is, one after another.
                                  This may be useful if you want to run bigger experiments without using too many system resources.
                                  Note that the use of this option \textbf{also} enables the \Opt{\ddash force} option.
                            \end{Description}

                    \item[\Arg{slurm}]
                        Runs the experiment on a \Prog{Slurm}-equipped cluster computer.

                        In this mode, \Prog{gourd} will use the \Prog{Slurm} command-line
                        interface to schedule runs on a supercomputer.
                        The prerequisites are that:
                        \begin{itemize}
                            \item \Prog{gourd} is running on the login node of a supercomputer, such
                                  that the \Prog{srun} command is available.
                            \item \File{gourd.toml} contains all required fields for running on
                                  \Prog{Slurm} (see the manual for \Prog{gourd.toml}(5))
                            \item \File{gourd.toml} contains a valid experiment for which all
                                  paths (including the programs and output paths) are accessible
                                  from the cluster nodes.
                        \end{itemize}

                        When \Prog{gourd} \Arg{run} \Arg{slurm} is called, the experiment's runs
                        are not executed immediately; instead, they are submitted as \emph{job arrays}
                        to the \Prog{Slurm} scheduler.
                        The experiment's runs are then in the supercomputer's queue (status ``pending'').
                        The time until the runs are actually executed depends on many factors, which
                        may include the current load and the size of your experiment; this
                        delay can range from seconds to days.
                        For this reason, \Prog{gourd} \Arg{run} \Arg{slurm} does not show the
                        continuous status of an experiment, use \Prog{gourd} \Arg{status} to do that.

                        On successful scheduling, the Slurm IDs of the job arrays that make
                        up the experiment will be shown, and the command will exit.
                        To view the experiment's status, see the \Prog{gourd} \Arg{status} section
                        of this manual.

                        Running on Slurm has many configurable options.
                        Please refer to the manual \Prog{gourd-tutorial}(1) for example setups
                        and the manual \Prog{gourd.toml}(5) for complete reference.
                        The implementation of the Slurm API used by \Prog{gourd} is discussed
                        in depth in the \Prog{gourd} maintainer documentation.
                \end{Description}

        \subsection{GOURD INIT}

            \subsubsection{Summary}
                The \Prog{gourd} \Arg{init} command creates an experimental configuration.
                Configurations are represented as TOML files.
                A template configuration, \File{gourd.toml}, is created in the directory specified.
                The directory can optionally be initialized as a Git repository.
                Unless run with the \oOpt{-s} flag, this command will ask using interactive prompts
                to refine the template to your needs.

                If the command is run with the \oOpt{-s} flag these choices will not be offered and
                the default options will be picked for all queries.

            \subsubsection{Synopsis}
                \Prog{gourd} \Arg{init}
                \oOpt{GLOBAL OPTIONS}
                \oOptArg{-e}{ example-name}
                \oOpt{\ddash list-examples}
                \oOptArg{\ddash git}{=true|false}
                \oArg{directory}

            \subsubsection{Options}
                \begin{Description}[Options]
                    \item[\OptArg{-e, \ddash example}{ example-name}]
                      Initializes the given directory with an example configuration from \Prog{gourd-tutorial(7)}
                      (rather than a custom template for \File{gourd.toml}).
                    \item[\Opt{\ddash list-examples}]
                      Instead of initializing a folder, this will make \Prog{gourd} list all the available
                      examples for the \emph{-e} option.
                    \item[\OptArg{\ddash git}{=true|false}]
                      Whether to initialize an empty git repository in the newly created folder.
                \end{Description}

            \subsubsection{Listing Examples}
                If \Opt{\ddash list-examples} is used, \Prog{gourd} \Arg{init} will not initialize a new folder with
                a configuration.
                The \Arg{directory} argument will be ignored.

                A list of available examples and their descriptions will be printed to the output and
                the program will exit.

        \subsection{GOURD STATUS}

            \subsubsection{Summary}
                The \Prog{gourd} \Arg{status} command displays the status of an existing experiment,
                that is, one that has been created by \Prog{gourd} \Arg{run}, but not necessarily
                one that has fully executed.
                This command can also display detailed status of an individual run using the \Opt{-i} flag.

            \subsubsection{Synopsis}
                \Prog{gourd} \Arg{status}
                \oOpt{GLOBAL OPTIONS}
                \oOptArg{-i}{ run-id}
                \oOpt{\ddash follow}
                \oOpt{\ddash full}
                \oOpt{\ddash after-out}
                \oArg{experiment-id}

            \subsubsection{Options}
                \begin{Description}[Options]
                  \item[\Arg{experiment-id}]
                    The ID of an experiment to show the status of.
                    By default, this is the most recent experiment.
                  \item[\OptArg{-i}{ run-id}]
                    Instead of showing a general overview of the entire experiment show detailed
                    information about a run with this \emph{run-id}.
                  \item[\Opt{\ddash full}]
                    By default, \Arg{status} displays a summary rather than a full list if there
                    is a large number of runs (>100). Using \Opt{\ddash full}, the full list is
                    always shown.
                  \item[\Opt{\ddash follow}]
                    The status will be continually displayed until all of the runs have finished.
                    This is useful when it is known that the jobs will finish
                    in a matter of minutes.
                  \item[\Opt{\ddash after-out}]
                      Use only with \OptArg{-i}{ run-id}, displays the raw afterscript output for that run.
                \end{Description}

            \subsubsection{Experiment status}
                By default, \Prog{gourd} \Arg{status} uses the \File{gourd.toml} file to determine the
                location of experiment files generated using \Prog{gourd} \Arg{run}.
                It finds the most recent experiment (unless \oArg{experiment-id} is specified) and shows a summary
                containing the status of each run, and, if completed, the run's basic timing metrics.
                The command also shows a summary of each run's error status, if any.

            \subsubsection{Run status}
                With the \OptArg{-i}{ run-id} argument, \Prog{gourd} \Arg{status} will retrieve detailed
                run information including the arguments that the binary was called with, RUsage metrics
                if successful, and detailed error status if it has failed.
                The file paths provided make it easy to inspect the output of a run, whether it has
                succeeded or failed.

            \subsubsection{Afterscripts}
                To postprocess the output of the runs, there are two options available: \emph{afterscipts} and \emph{pipelining}.
                Afterscripts are scripts that run locally (so for DelftBlue they do
                \emph{not} get scheduled as separate jobs).

                Afterscripts are meant for quick and computationally inexpensive postprocessing 
                (such as getting the first line of the output file).
                For long or complicated postprocessing with a significant computational cost, look at \emph{Pipelining}.

                \begin{itemize}
                    \item An afterscript is optional and specified per program. 
                    \item To indicate the use of an afterscript, the path to the script file needs to be specified in the \File{gourd.toml} under the chosen program.
                    \item Multiple programs can use the same script. 
                    \item The afterscript can be used to assign labels to runs as a means of specifying custom status.
                \end{itemize}

                \emph{How to design an afterscript:}

                The afterscript should be an \underline{executable} file. This can be a normal compiled executable, or possibly a shell/python script if you use the appropriate \emph{shebang} at the start of the file (check out https://en.wikipedia.org/wiki/Shebang\_(Unix) for details).

                \Prog{gourd} will pass the path to a file containing the main program's output to the afterscript as a command line argument.
                The afterscript can then print any output to \texttt{stdout} 
                (via \texttt{print}, \texttt{printf}, \texttt{println}, \texttt{echo} or your preferred language's method),
                and \Prog{gourd} will collect that and display it in 
                \Prog{gourd}~\Arg{status}~\Arg{-i}~\Arg{<specific run id>}
                
                \emph{What can you do with afterscript output}

                \begin{itemize}
                    \item use \Prog{labels}, check out the corresponding chapter for more details.
                    \item create custom metrics: read in \Prog{gourd}~\Arg{analyse} for how to do this.
                \end{itemize}
        \subsection{GOURD CONTINUE}

            \subsubsection{Summary}
                The \Prog{gourd} \Arg{continue} command schedules runs that are part of an existing
                experiment, but have not yet been scheduled.
                For example, an experiment with 30,000 distinct runs can be scheduled in three batches
                of 10,000 each if that is the maximum number of queued supercomputer jobs.

            \subsubsection{Synopsis}
                \Prog{gourd} \Arg{continue}
                \oOpt{GLOBAL OPTIONS}
                \oArg{experiment-id}

            \subsubsection{Options}
                \begin{Description}[Options]
                  \item[\Arg{experiment-id}]
                    The ID of an experiment to continue.
                    By default, this is the most recent experiment.
                \end{Description}

            \subsubsection{Postprocessing Slurm jobs}
                As discussed when describing \Prog{gourd} \Arg{status}, to postprocess the output of the runs,
                there are two options available: afterscipts and Slurm postprocessing jobs. Afterscripts are
                scripts that run locally (so for DelftBlue they do not get scheduled as separate jobs) and are
                thus meant for quick and non-complicated postprocessing (such as getting the first line of the
                output file). For long and complicated postprocessing with a significant computational cost,
                we support Slurm postprocessing jobs. A program being evaluated can have both an afterscript
                and a postprocessing Slurm job, one of them, or neither.

                A postprocesing Slurm job (further called ``postprocessing job'') is optional and specified per
                program. To indicate the use of a postprocessing job, a program needs to be specified under
                ``postprocessing programs'' in gourd.toml. That program will have the path to the postprocessing
                binary file. In addition, the name of this new postprocesing program needs to be specified in
                the gourd.toml under the chosen regular program to indicate that this is the postprocessing
                used. Multiple programs can use the same postprocessing program. Furthermore, if at least one
                program has a postprocessing job, a path to a folder that will store the postprocesing job
                output needs to be specified (once for the entire experiment, analogous to metrics and output
                paths).

                As input the postprocessing binary will get the output of a run of a regular program that has
                been specified to have this postprocessing. It will write its results to a file the way that
                regular programs do.

        \subsection{GOURD CANCEL}

            \subsubsection{Summary}
                The \Prog{gourd} \Arg{cancel} command cancels runs that have been scheduled on Slurm.
                By default, it cancels all scheduled runs in the most recent experiment.
                This command can cancel an individual run using the \Opt{-i} flag.

            \subsubsection{Synopsis}
                \Prog{gourd} \Arg{cancel}
                \oOpt{GLOBAL OPTIONS}
                \oArg{experiment-id}
                \oOptArg{-i}{ run-ids}
                \oOpt{-a}

            \subsubsection{Options}
                \begin{Description}[Options]
                  \item[\Arg{experiment-id}]
                    The ID of an experiment to cancel runs from.
                    By default, this is the most recent experiment.
                  \item[\OptArg{-i}{ run-ids}]
                    The IDs of the runs to cancel.
                    Pass multiple run IDs separated by spaces, for example \Arg{-i 1 2 3}.
                    By default, all runs in the experiment are cancelled.
                  \item[\Opt{-a, \ddash all}]
                    Cancel all runs from this account.
                    This includes all runs, not just those from \Prog{gourd}.
                \end{Description}

            \subsubsection{Cancelling All Runs}
                Cancelling all runs will \textbf{cancel all runs scheduled on your account}.
                This option is included to be able to cancel past or deleted experiments.
                But be aware of its possible impact.

                You can see which runs would be cancelled without actually doing it
                by running \Prog{gourd} \Arg{cancel} \Arg{\ddash all} \Arg{\ddash dry}.

            \subsubsection{Latency}
                Slurm may take some time to acknowledge the cancellation; thus, running
                \Prog{gourd} \Arg{status} right away after a cancellation may still display
                the runs as pending, please wait up to one minute for the changes to propagate.


        \subsection{GOURD ANALYSE}

            \subsubsection{Summary}
                The \Prog{gourd} \Arg{analyse} command collects and processes metrics generated
                when an experiment was run. It can produce a CSV data file or a ``cactus plot''
                to compare how quickly different algorithms run.

            \subsubsection{Synopsis}
                \Prog{gourd} \Arg{analyse}
                \oOpt{GLOBAL OPTIONS}
                \oOptArg{-o}{ format}
                \oArg{experiment-id}

            \subsubsection{Options}
                \begin{Description}[Options]
                  \item[\Arg{experiment-id}]
                    The ID of an experiment to analyse.
                    By default, this is the most recent experiment.
                  \item[\OptArg{-o}{ format}, \OptArg{\ddash output}{ format}]
                    The format of the desired analysis output. There are three available:
                    \emph{csv} (the default), \emph{plot-png}, \emph{plot-svg}. They are described below.
                \end{Description}

            \subsubsection{Metrics CSV}
                Running \Prog{gourd} \Arg{analyse} \OptArg{-o}{ csv} will create a CSV file with
                data about the status of the runs, metrics, and afterscript completion, unless there
                are no runs that have completed already.
                The CSV generation will take into account all runs of the experiment. If \Prog{gourd}
                \Arg{analyse} is rerun, the CSV will be updated with the newest status of the runs.

            \subsubsection{Cactus plots}
                Running \Prog{gourd} \Arg{analyse} \OptArg{-o}{ plot-png} will create a PNG picture of
                a cactus plot.
                The cactus plot is used to showcase the comparison of how many inputs each algorithm
                can finish running with in a given amount of time.
                In other words, the horizontal axis represents the time passing, and the vertical axis
                represents how many runs of this program (algorithm) have already finished.
                This allows to see a visual comparison of the time each program takes - the more runs
                there are, the more informative the plot will result to be.
                The plot will take into account only the runs that have completed and have valid
                RUsage data. If \Prog{gourd} \Arg{analyse} is rerun, the graph will be updated
                according to the newest available data.

                Running \Prog{gourd} \Arg{analyse} \OptArg{-o}{ plot-svg} will create exactly the same
                plot but in a \emph{svg} conformant format.

        \subsection{GOURD SET-LIMITS}

            \subsubsection{Summary}
                The \Prog{gourd} \Arg{set-limits} command allows to update resource limits for programs.
                The new resource limits will affect future runs that have these programs.

            \subsubsection{Synopsis}
                \Prog{gourd} \Arg{set-limits}
                \oOpt{GLOBAL OPTIONS}
                \oOptArg{-p}{ program-name}
                \oOpt{-a}
                \oOptArg{--mem}{ memory-limit}
                \oOptArg{--cpu}{ cpu-limit}
                \oOptArg{--time}{ time-limit}
                \oArg{experiment-id}

            \subsubsection{Options}
                \begin{Description}[Options]
                    \item[\Arg{experiment-id}]
                    The ID of an experiment to modify limits of.
                    By default, this is the most recent experiment.
                    \item[\Opt{-a, \ddash all}]
                    Changes resource limits for all available programs, including postprocessing programs.
                    \item[\OptArg{-p, \ddash program}{ program-name}]
                    Changes resource limits for the specified program (simple or postprocesing).
                    \item[\OptArg{\ddash mem}{ memory-limit}]
                    Allows to specify the new memory limit (a number) instead of asking for it interactively.
                    \item[\OptArg{\ddash cpu}{ cpu-limit}]
                    Allows to specify the new cpu limit (a number) instead of asking for it interactively.
                    \item[\OptArg{\ddash time}{ time-limit}]
                    Allows to specify the new time limit (in humantime format) instead of asking for it
                    interactively.
                \end{Description}

            \subsubsection{Programs}
                Specifying a program to modify the limits for is required.
                One can either specify the \Opt{-p} flag with a specific program
                or \Opt{-a} to modify limits for all programs

            \subsubsection{Scripting}
                When ran with \Opt{-s}, \Arg{set-limits} will not ask for the properties
                not specified by command line options, instead it will exit with an error.

        \subsection{GOURD VERSION}

            \subsubsection{Summary}
                \Prog{gourd} \Arg{version} outputs the software version and exits.
                Using the \oOpt{-s} flag will display the version only, otherwise \Prog{gourd}
                will stress-test your terminal font.

            \subsubsection{Synopsis}
                \Prog{gourd} \Arg{version} \oOpt{-s}

            \subsubsection{Scripting}
                By default \Prog{gourd} \Arg{version} shows a human readable only output. By
                running \Prog{gourd} \Arg{version} \Opt{-s} one can obtain a version number
                in the format:

                \texttt{gourd <version number>}

    \section{EXAMPLES}

        See the section on \Prog{gourd} \Arg{init} for runnable example directories.
        For a more detailed walkthrough with more focus on examples, use the \Prog{gourd-tutorial(7)} manual.

    \section{FILES}

        \begin{Description}[Files]\setlength{\itemsep}{0cm}
            \item[\File{gourd.toml}] A configuration file containing the experiment details. See \Prog{gourd.toml(5)}.
            \item[\File{<experiment-dir>/<experiment-number>.lock}] A file containing the runtime data of the experiment.
        \end{Description}

    \section{SEE ALSO}

        \Prog{gourd-tutorial(7)}

        \Prog{gourd.toml(5)}

    \section{AUTHORS}
    Rūta Giedrytė <\Email{r.giedryte@student.tudelft.nl}>\\[0.1cm]\MANbr
    Lukáš Chládek <\Email{l@chla.cz}>\\[0.1cm]\MANbr
    Jan Piotrowski <\Email{me@jan.wf}>\\[0.1cm]\MANbr
    Mikołaj Gazeel <\Email{m.j.gazeel@student.tudelft.nl}>\\[0.1cm]\MANbr
    Ανδρέας Τσατσάνης <\Email{a.tsatsanis@student.tudelft.nl}>\\[0.1cm]\MANbr

%@% IF LATEX %@%
\end{adjustwidth}
%@% END-IF %@%


\end{document}
