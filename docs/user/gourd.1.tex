\documentclass[a4paper,english]{article}
\usepackage{a4wide}
\usepackage{babel}
\usepackage{verbatim}

\usepackage{changepage}

\usepackage[bookmarksopen,bookmarksnumbered]{hyperref}

\usepackage[fancyhdr]{latex2man}

\usepackage{xspace}



\newcommand{\thecmd}{gourd}
\newcommand{\thecommand}{GOURD}
\newcommand{\mansection}{1}
\newcommand{\mansectionname}{DelftBlue Tools Manual}
\newcommand{\mandate}{5 MAY 2024}
\setDate{5 MAY 2024}
\setVersionWord{Version:}
\setVersion{1.29}


\fancyhead[L, R]{\textit{\thecommand}(\mansection)}
\fancyhead[C]{\textsc{\mansectionname}}
\fancyfoot[C]{\mandate}
\fancyfoot[R]{\thepage}
\renewcommand{\headrulewidth}{0pt}

\renewcommand{\Prog}[1]{\textbf{#1}}                      % Program name

\renewenvironment{abstract}{\noindent\bfseries NAME\normalfont\vspace{0.2cm}}{}

\renewenvironment{Name}[5]{
\title{#5}
\author{#3}
\date{\@LM@Date\\{\small Version \@LM@Version}}
\begin{abstract}
}{
\end{abstract}
\vspace{-0.8cm}
}

\renewenvironment{Description}[1][]{
\begin{list}{}{
 \ifthenelse{\equal{#1}{}}{
   % optional argument not given
     \labelwidth\z@ \itemindent-\leftmargin
     \let\makelabel\descriptionlabel
     \renewcommand{\makelabel}[1]{\hspace\labelsep\normalfont\bfseries##1}
  }{
   % optional argument given
   \settowidth{\labelwidth}{\normalfont\bfseries#1}
   \setlength{\leftmargin}{\labelwidth}
   \addtolength{\leftmargin}{\labelsep}
   \renewcommand{\makelabel}[1]{\normalfont\bfseries##1\hfil}
  }}
}{
\end{list}
}

\usepackage{titlesec}
\titleformat{\section}
  {\normalfont\normalsize\bfseries}{\thesection}{0.5em}{}

\titleformat{\subsection}
  {\normalfont\normalsize\bfseries}{\thesubsection}{0.5em}{}

\setlength{\parindent}{0pt}
\setlength{\parskip}{6pt}

% Improve the \section command if in a LaTeX environment.
%@% IF LATEX %@%
\let\oldsection\section
\renewcommand{\section}[1]{%
\end{adjustwidth}%
\oldsection*{#1}%
\begin{adjustwidth}{18pt}{0pt}%
}
\let\oldsubsection\subsection
\renewcommand{\subsection}[1]{%
\oldsubsection*{#1}%
}
\let\oldsubsubsection\subsubsection
\renewcommand{\subsubsection}[1]{%
\oldsubsubsection*{#1}%
}

\newcommand{\ddash}{-{}-}
%@% END-IF %@%


\usepackage{mathspec}
\setmainfont[Mapping=tex-text, FakeBold=1]{Linux Libertine O}
\setmathfont(Digits,Greek,Latin)[Numbers=OldStyle, FakeBold=1]{Linux Libertine O}

\begin{document}
    \pagestyle{fancy}


    \begin{Name}{1}{gourd}{gourd}{DelftBlue Tools Manual}{Gourd}
%@% IF LATEX %@%
\begin{adjustwidth}{18pt}{0pt}
%@% END-IF %@%

        \Prog{gourd} - A tool for scheduling parallel runs for algorithm comparisons.

%@% IF LATEX %@%
\end{adjustwidth}
%@% END-IF %@%
    \end{Name}


%@% IF LATEX %@%
\begin{adjustwidth}{18pt}{0pt}
%@% END-IF %@%
    \section{SYNOPSIS}

        \Prog{gourd} \Arg{command}
        \oOptArg{-c}{ filename}
        \oOpt{-d}
        \oOpt{-h}
        \oOpt{-s}
        \oOpt{-v|-vv}

    \section{DESCRIPTION}

        \Prog{gourd} is a tool that schedules parallel runs for algorithm comparisons.
        Given the parameters of the experiment, a number of test datasets, and algorithm implementations to compare,
        \Prog{gourd} runs the experiment in parallel and provides many options for processing its results.
        While originally envisioned for the DelftBlue supercomputer at Delft University of Technology,
        \Prog{gourd} can replicate the experiment on any cluster computer with the \Prog{Slurm} scheduler,
        on any UNIX-like system, and on Microsoft Windows.

        New to \Prog{gourd}? Go directly to the \Prog{gourd-tutorial(7)} manual.


    \section{OPTIONS}

        The following options apply to all \Prog{gourd} commands.

        \begin{Description}[Options]
            \item[\OptArg{-c}{ filename}, \OptArg{--config}{ filename}]
            Tell \Prog{gourd} to use the given filename as \File{gourd.toml}, the configuration
            file that defines the experimental setup.
            By default, the file is expected in the current working directory at \File{./gourd.toml}.
            \item[\Opt{-d}, \Opt{--dry-run}]
            Run \Prog{gourd} in dry-run mode, printing all operations (such as writing to files or scheduling runs)
            without executing them.
            \item[\Opt{-h}, \Opt{--help}]
            Display usage instructions for the \Prog{gourd} utility or any of its commands.
            \item[\Opt{-s}, \Opt{--script}]
            Tell \Prog{gourd} to use a script-friendly interface, that is, one that does not use
            interactive user prompts.
            \item[\Opt{-v}, \Opt{-vv}, \Opt{--verbose}]
            Run \Prog{gourd} with verbose output, where \Opt{-vv} enables even more logging.
        \end{Description}

    \section{COMMANDS}

        Using \Prog{gourd} is as simple as invoking one of its commands, such as
        \Prog{gourd}~\Arg{status}.
        Command-line arguments are generally not necessary; to design and run
        experiments, a \File{gourd.toml} file should be in the current directory.
        The following is a summary of available commands.

        \begin{Description}[Commands]
            \item[\Prog{gourd} \Arg{run}]
            Create an experiment from configuration and run it on \Prog{Slurm} or the local machine.
            \item[\Prog{gourd} \Arg{init}]
            Set up a template of an experiment configuration.
            \item[\Prog{gourd} \Arg{status}]
            Display the status of a running or completed experiment.
            \item[\Prog{gourd} \Arg{continue}]
            Schedule the incomplete part of a partial experiment.
            \item[\Prog{gourd} \Arg{cancel}]
            Cancel scheduled runs.
            \item[\Prog{gourd} \Arg{analyse}]
            Output metrics of completed runs.
            \item[\Prog{gourd} \Arg{version}]
            Show the software version.
        \end{Description}

        \subsection{GOURD RUN}

            \subsubsection{Summary}
                The \Prog{gourd} \Arg{run} command uses the provided configuration and runs an
                experiment.
                Using either \Arg{local} or \Arg{slurm}, it is possible for the execution
                to run on the local machine, or be scheduled using Slurm on a cluster computer.
                Using Slurm, additional configuration arguments are required; see
                \Prog{gourd.toml(5)}.

                In principle, however, using \Prog{gourd}~\Arg{run} works in the same way for both
                Slurm and local execution.
                A \File{gourd.toml} configuration file should be present in the current directory,
                formally describing the experiment that is to be created and run.
                Because most options are specified in this file, it is usually sufficient to type
                \Prog{gourd}~\Arg{run}~\Arg{slurm}|\Arg{local} to run an experiment.

                See the manual page for \Prog{gourd-tutorial(7)} for a step-by-step guide on
                designing experiments to run.


            \subsubsection{Synopsis}
                \Prog{gourd}
                \Arg{run}
                \Arg{slurm}|\Arg{local}
                \oOptArg{-c}{ filename}
                \oOpt{-f}
                \oOpt{-h}
                \oOpt{-s}
                \oOpt{-v|-vv}

            \subsubsection{Subcommands}
                \begin{Description}[Subcommands]
                    \item[\Arg{local}]
                        Runs the experiment locally, such that all programs are executed directly on the computer
                        that \Prog{gourd} is run from.

                        This is useful for running \textbf{small parts} of an experiment on
                        a personal computer or a login node, allowing you to test that programs
                        are being called correctly and that the configuration is valid.
                        Please note that \emph{local} is \textbf{NOT} intended for running full
                        experiments on a \Prog{Slurm}-equipped cluster.
                        \emph{Local} will use the login node only and not the actual
                        supercomputer.

                        Running using \Arg{run}~\Arg{local} will perform the experiments in
                        parallel based on the number of available CPU cores.
                        Resource limits set in the configuration will not be honoured.
                        While running, experiment status is displayed continuously (see the
                        \Prog{gourd} \Arg{status} command) until all runs have finished executing.
                        Typing Control+C into the terminal will stop the runs.

                    \item[\Arg{slurm}]
                        Runs the experiment on a \Prog{Slurm}-equipped cluster computer.

                        In this mode, \Prog{gourd} will use the \Prog{Slurm} command-line
                        interface to schedule runs on a supercomputer.
                        The prerequisites are that:
                        \begin{itemize}
                            \item \Prog{gourd} is running on the login node of a supercomputer, such
                                  that the \Prog{srun} command is available.
                            \item \File{gourd.toml} contains all required fields for running on
                                  \Prog{Slurm} (see the manual for \Prog{gourd.toml}(5))
                            \item \File{gourd.toml} contains a valid experiment for which all
                                  paths (including the programs and output paths) are accessible
                                  from the cluster nodes.
                        \end{itemize}

                        When \Prog{gourd}~\Arg{run}~\Arg{slurm} is called, the experiment's runs
                        are not executed immediately; instead, they are submitted as \emph{job arrays}
                        to the \Prog{Slurm} scheduler.
                        The experiment's runs are then in the supercomputer's queue (status ``pending'').
                        The time until the runs are actually executed depends on many factors, which
                        may include the supercomputer load and the size of your experiment; this
                        delay can range from seconds to days.
                        For this reason, \Prog{gourd}~\Arg{run}~\Arg{slurm} does not show the
                        continuous status of an experiment unless the \oOpt{-f,~--follow} flag is
                        used.

                        On successful scheduling, the Slurm IDs of the job arrays that make
                        up the experiment will be shown, and the command will exit.
                        To view the experiment's status, see the \Prog{gourd}~\Arg{status} section
                        of this manual.

                        Running on Slurm has many configurable options.
                        Please refer to the manual for \Prog{gourd-tutorial}(1) for example setups
                        and the manual for \Prog{gourd.toml}(5) for complete reference.
                        The implementation of the Slurm API used by \Prog{gourd} is discussed
                        in depth in the \Prog{gourd} maintainer documentation.
                \end{Description}


        \subsection{GOURD INIT}

            \subsubsection{Summary}
                The \Prog{gourd} \Arg{init} command creates an experimental configuration.
                Configurations are represented as TOML files.
                A template configuration, \File{gourd.toml}, is created in the directory specified.
                The directory can optionally be initialized as a Git repository.
                Unless run with the \oOpt{-q} flag, this command will ask using interactive prompts
                to refine the template to your needs.

            \subsubsection{Synopsis}
                \Prog{gourd} \Arg{init}
                \oOpt{-d}
                \oOptArg{-e}{ example-name}
                \oOpt{-h}
                \oOpt{-s}
                \oOpt{-v|-vv}
                \oArg{directory}

            \subsubsection{Options}
                \begin{Description}[Options]
                \item[\OptArg{-e, --example}{ example-name}]
                Initializes the given directory with an example configuration from \Prog{gourd-tutorial(7)}
                (rather than a custom template for \File{gourd.toml}).
                \item[\Opt{-q, --quick}]
                Does not give interactive prompts, using default values to create a minimal \File{gourd.toml}.
                This is equivalent to using \Opt{-s}.
                \end{Description}

        \subsection{GOURD STATUS}

            \subsubsection{Summary}
                The \Prog{gourd}~\Arg{status} command displays the status of an existing experiment,
                that is, one that has been created by \Prog{gourd} \Arg{run}, but not necessarily
                one that has fully executed.
                This command can also display detailed status of an individual run using the \Opt{-i} flag.

            \subsubsection{Synopsis}
                \Prog{gourd}\Arg{status}
                \oArg{experiment-id}
                \oOpt{-d}
                \oOpt{-h}
                \oOptArg{-i}{ run-id}
                \oOpt{-s}
                \oOpt{-v|-vv}

            \subsubsection{Options}
                \begin{Description}[Options]
                \item[\Arg{experiment-id}]
                The ID of an experiment to show the status of.
                By default, this is the most recent experiment.
                \item[\OptArg{-i}{ run-id}]
                The ID of a run to show detailed information of.
                \end{Description}

            \subsection{Experiment status}
                By default, \Prog{gourd}~\Arg{status} uses the \File{gourd.toml} file to determine the
                location of experiment files generated using \Prog{gourd}~\Arg{run}.
                It finds the most recent experiment (unless \Opt{-i} is specified) and shows a summary
                containing the status of each run, and, if completed, the run's basic timing metrics.
                The command also shows a summary of each run's error status, if any.

            \subsection{Run status}
                With the \OptArg{-i}{ run-id} argument, \Prog{gourd}~\Arg{status} will retrieve detailed
                run information including the arguments that the binary was called with, RUsage metrics
                if successful, and detailed error status if it has failed.
                The file paths provided make it easy to inspect the output of a run, whether it has
                succeeded or failed.

        \subsection{GOURD CONTINUE}

            \subsubsection{Summary}
                The \Prog{gourd} \Arg{continue} command schedules runs that are part of an existing
                experiment, but have not yet been scheduled.
                For example, an experiment with 30,000 distinct runs can be scheduled in three batches
                of 10,000 each if that is the maximum number of queued supercomputer jobs.


%            \subsubsection{Synopsis}
%                \emph{TBD.}
%
%            \subsubsection{Options}
%                \emph{TBD.}


    \subsection{GOURD CANCEL}

        \subsubsection{Summary}
            The \Prog{gourd}~\Arg{cancel} command cancels runs that have been scheduled on \Prog{Slurm}.
            By default, it cancels all scheduled runs in the most recent experiment.
            This command can cancel an individual run using the \Opt{-i} flag.

        \subsubsection{Synopsis}
            \Prog{gourd}\Arg{cancel}
            \oArg{experiment-id}
            \oOptArg{-i}{ run-ids}
            \oOpt{-a}
            \oOpt{-v|-vv}

        \subsubsection{Options}
            \begin{Description}[Options]
                \item[\Arg{experiment-id}]
                The ID of an experiment to cancel.
                By default, this is the most recent experiment.
                \item[\OptArg{-i}{ run-ids}]
                The IDs of the runs to cancel.
                Pass multiple run IDs separated by spaces, for example \Arg{-i 1 2 3}.
                By default, all runs in the experiment are cancelled.
                \item[\oOpt{--all}]
                Cancel all runs from this account.
                This includes all runs, not just those from \Prog{gourd}.
                To see what runs would be cancelled, run \Prog{gourd}~\Arg{cancel}~\Arg{--all}~\Arg{--dry}
            \end{Description}


        \subsection{GOURD ANALYSE}

            \subsubsection{Summary}
                The \Prog{gourd} \Arg{analyse} command collects and processes metrics generated
                when an experiment was run.
                The command is not yet implemented.

        \subsection{GOURD VERSION}

            \subsubsection{Summary}
                \Prog{gourd} \Arg{version} outputs the software version and exits.
                Using the \oOpt{-s} flag will display the version only, otherwise \Prog{gourd}
                will stress-test your terminal font.

            \subsubsection{Synopsis}
                \Prog{gourd} \Arg{version} \oOpt{-s}



    \section{EXAMPLES}

        This section contains some common examples of using \Prog{gourd}.
        For a more detailed walkthrough with more focus on examples, use the \Prog{gourd-tutorial(7)} manual.




    \section{FILES}

        \begin{Description}[Files]\setlength{\itemsep}{0cm}
            \item[\File{gourd.toml}] A configuration file containing the experiment details. See \Prog{gourd.toml(5)}.
            \item[\File{<experiment-dir>/<experiment-number>.lock}] A file containing the runtime data of the experiment.
        \end{Description}

    \section{SEE ALSO}

        \Prog{gourd-tutorial(7)}

        \Prog{gourd.toml(5)}

    \section{AUTHORS}
    Rūta Giedrytė <\Email{r.giedryte@student.tudelft.nl}>\\[0.1cm]\MANbr
    Lukáš Chládek <\Email{l@chla.cz}>\\[0.1cm]\MANbr
    Jan Piotrowski <\Email{j.p.piotrowski@student.tudelft.nl}>
    Mikołaj Gazeel <\Email{m.j.gazeel@student.tudelft.nl}>\\[0.1cm]\MANbr
    Ανδρέας Τσατσάνης <\Email{a.tsatsanis@student.tudelft.nl}>\\[0.1cm]\MANbr
%@% IF LATEX %@%
\end{adjustwidth}
%@% END-IF %@%


\end{document}
