\documentclass[a4paper,english]{article}
\usepackage{a4wide}
\usepackage{babel}
\usepackage{verbatim}

\usepackage{changepage}

\usepackage[bookmarksopen,bookmarksnumbered]{hyperref}

\usepackage[fancyhdr]{latex2man}



\newcommand{\thecmd}{gourd.toml}
\newcommand{\thecommand}{GOURD.TOML}
\newcommand{\mansection}{1}
\newcommand{\mansectionname}{File Formats Manual}
\newcommand{\mandate}{5 MAY 2024}
\setDate{5 MAY 2024}
\setVersionWord{Version:}
\setVersion{1.29}


\fancyhead[L, R]{\textit{\thecommand}(\mansection)}
\fancyhead[C]{\textsc{\mansectionname}}
\fancyfoot[C]{\mandate}
\fancyfoot[R]{\thepage}
\renewcommand{\headrulewidth}{0pt}

\renewcommand{\Prog}[1]{\textbf{#1}}                      % Program name

\renewenvironment{abstract}{\noindent\bfseries NAME\normalfont\vspace{0.2cm}}{}

\renewenvironment{Name}[5]{
\title{#5}
\author{#3}
\date{\@LM@Date\\{\small Version \@LM@Version}}
\begin{abstract}
}{
\end{abstract}
\vspace{-0.8cm}
}

\renewenvironment{Description}[1][]{
\begin{list}{}{
 \ifthenelse{\equal{#1}{}}{
   % optional argument not given
     \labelwidth\z@ \itemindent-\leftmargin
     \let\makelabel\descriptionlabel
     \renewcommand{\makelabel}[1]{\hspace\labelsep\normalfont\bfseries##1}
  }{
   % optional argument given
   \settowidth{\labelwidth}{\normalfont\bfseries#1}
   \setlength{\leftmargin}{\labelwidth}
   \addtolength{\leftmargin}{\labelsep}
   \renewcommand{\makelabel}[1]{\normalfont\bfseries##1\hfil}
  }}
}{
\end{list}
}

\usepackage{titlesec}
\titleformat{\section}
  {\normalfont\normalsize\bfseries}{\thesection}{0.5em}{}

\titleformat{\subsection}
  {\normalfont\normalsize\bfseries}{\thesubsection}{0.5em}{}

\setlength{\parindent}{0pt}
\setlength{\parskip}{6pt}

% Improve the \section command if in a LaTeX environment.
%@% IF LATEX %@%
\let\oldsection\section
\renewcommand{\section}[1]{%
\end{adjustwidth}%
\oldsection*{#1}%
\begin{adjustwidth}{18pt}{0pt}%
}
\let\oldsubsection\subsection
\renewcommand{\subsection}[1]{%
\oldsubsection*{#1}%
}
\let\oldsubsubsection\subsubsection
\renewcommand{\subsubsection}[1]{%
\oldsubsubsection*{#1}%
}

\newcommand{\ddash}{-{}-}
%@% END-IF %@%


\usepackage{mathspec}
\setmainfont[Mapping=tex-text, FakeBold=1]{Linux Libertine O}
\setmathfont(Digits,Greek,Latin)[Numbers=OldStyle, FakeBold=1]{Linux Libertine O}

\begin{document}
  \pagestyle{fancy}

  \begin{Name}{5}{gourd.toml}{Test}{File Formats Manual}{gourd.toml}
%@% IF LATEX %@%
\begin{adjustwidth}{18pt}{0pt}
%@% END-IF %@%

      \Prog{gourd.toml} \- An experimental setup file for the Gourd experiment scheduler.

%@% IF LATEX %@%
\end{adjustwidth}
%@% END-IF %@%
  \end{Name}

%@% IF LATEX %@%
\begin{adjustwidth}{18pt}{0pt}
%@% END-IF %@%

  \section{DESCRIPTION}

      \noindent \Prog{gourd.toml} is a declarative configuration file that,
      when provided to \Prog{gourd(1)}, can be used to create a new \emph{experiment}.
      The rules for typesetting the file follow the standard TOML format.

      By default, the filename is \File{gourd.toml}.

  \section{VALUE TYPES}

    \begin{Description}[Types]\setlength{\itemsep}{0cm}
        \item[string] A string of characters surrounded by "".
        \item[path] A file path surrounded by "".
        \item[list of T] A list of values of type T surrounded by [].
    \end{Description}

  \section{GLOBAL CONTEXT}

    These should be put at the beginning of the file before defining any sections.
    Options surrounded by brackets are optional and can be omitted, the default value is show in brackets in the description.

    \begin{Description}[Options]\setlength{\itemsep}{0cm}
        \item[\Opt{output\_path} = path]
            Where to store the stdout files from the \emph{programs}.
        \item[\Opt{metrics\_path} = path]
          Where to store the metrics for \Prog{gourd(1)} \Arg{status}.
        \item[\Opt{experiments\_folder} = path]
            Where to store state of previously ran experiments.
        \item[\Opt{wrapper?} = string]
            Defines the path to the \File{gourd-wrapper} binary.
            The default value is \Arg{gourd-wrapper}.
    \end{Description}


%    \section{SLURM}
%
%      An optional subsection of the config file, beginning with the \Arg{[slurm]} header.
%
%      \begin{Description}[Options]\setlength{\itemsep}{0cm}
%          \item[\Opt{experiment\_name} = string] The SLURM internal experiment name.
%          \item[TODO] TODO.
%      \end{Description}


    \section{PROGRAMS}

      Multiple programs can be specified.
      A program represents a compiled algorithm and is a combination of a binary file and parameters.
      Each program begins with \Arg{[programs.program-name]}, where \Arg{program-name} can be any unique name.

      \begin{Description}[Options]\setlength{\itemsep}{0cm}
          \item[\Opt{binary} = path]
            Path to the program executable.
          \item[\Opt{arguments?} = list of string]
            Arguments to be passed to the executable.
            The default is no arguments.
          \item[\Opt{afterscript?} = see afterscript]
            The configuration for running a short script after program completion.
      \end{Description}

    \section{INPUTS}

      A \Prog{gourd(1)} experiment consists of a cross-product mapping between programs
      and inputs.
      The experiment created from a \File{gourd.toml} file runs every combination
      of program and input in the file.

      Multiple inputs can be specified.
      Each input begins with \Arg{[inputs.input-name]} where \Arg{input-name} can be any unique name.
      \textbf{The string} \Arg{\_glob\_} \textbf{is reserved and cannot be used.}
      Each input contains the following keys:

      \begin{Description}[Options]\setlength{\itemsep}{0cm}
          \item[\Opt{input?} = path]
            Path to a file, the contents of which are passed to the program as standard input.
            By default, no standard input is provided.
          \item[\Opt{arguments?} = list of string]
            Additional command-line arguments to be passed to the program.
            The input arguments are appended to the programs arguments.
            By default, no additional arguments.
      \end{Description}


      \subsection{GLOBS}

        Globs can be applied to arguments and conveniently reference multiple files.

          If an argument starts with \Arg{glob|}, it will be treated as a glob.
          The input will be instantiated for every match of the provided glob.

            Example:

      \begin{verbatim}
[inputs.testrun1]
arguments = ["-f1", "=glob=./inputs/*.in", "-f2", "=glob=./input2/*.in"]
      \end{verbatim}

            Given that the current directory contains the files
            \File{input/1.in}, \File{input/2.in}, \File{input2/test.in}, the glob
            expands to the following experiment inputs:

      \begin{verbatim}
[inputs.testrun1_glob_0]
arguments = ["-f1", "./inputs/1.in", "-f2", "./input2/test.in"]

[inputs.testrun1_glob_1]
arguments = ["-f1", "./inputs/2.in", "-f2", "./input2/test.in"]
      \end{verbatim}

      \subsection{PARAMETERS}

        Parameters can be applied to the arguments to conveniently perform experiments with grid search.

            If an argument starts with \Arg{param|}, it will be treated as a parameter.
            For each value of that parameter the new input will be created with that value inserted into the argument
            into that argument place.

            Values of a parameter are specified in \Arg{[parameter.name]} using \Arg{values = list of string}

            This results in cross product between all parameters.

                Example:

        \begin{verbatim}
[inputs.testrun1]
arguments = ["-f", "param|x", "-x", "param|y"]

[parameters.x]
values = ["a", "b"]

[parameters.y]
values = ["10", "20"]
        \end{verbatim}

                It will be transformed into following inputs:

        \begin{verbatim}
[inputs.testrun1_x_0_y_0]
arguments = ["-f", "a", "-x", "10"]

[inputs.testrun1_x_0_y_1]
arguments = ["-f", "a", "-x", "20"]

[inputs.testrun1_x_1_y_0]
arguments = ["-f", "b", "-x", "10"]

[inputs.testrun1_x_1_y_1]
arguments = ["-f", "b", "-x", "20"]

[parameters.x]
values = ["a", "b"]

[parameters.y]
values = ["10", "20"]
        \end{verbatim}

    \subsubsection{SUBPARAMETERS}

        Subparameters are used when there is a need for 1-1 relation between two parameters.
        There is no cross product between subparameters of the same parameter.

        Subparameters are specified in inputs similarly to parameters with a difference of doing \Arg{subparam|parameter-name.subparameter-name}.

        Values of Subparameter are specified in \Arg{[parameter.name.sub.subparameter-name]} using \Arg{values = []}

        Note! Parameters can have either values or subparameters with values. Never both.

        Example:

        \begin{verbatim}
[input.testrun1]
arguments = ["-f", "subparam|x.1", "-x", "param|y", "-g", "subparam|x.2"]

[parameter.x.sub.1]
values = ["a", "b"]

[parameter.x.sub.2]
values = ["c", "d"]

[parameter.y]
values = ["10", "20"]
        \end{verbatim}

                It will be transformed into following inputs:

        \begin{verbatim}
[input.testrun1_x-0_y-0]
arguments = ["-f" "a", "-x", "10", "-g", "c"]

[input.testrun1_x-0_y-1]
arguments = ["-f", "a", "-x", "20", "-g", "c"]

[input.testrun1_x-1_y-0]
arguments = ["-f", "b", "-x", "10", "-g", "d"]

[input.testrun1_x-1_y-1]
arguments = ["-f", "b", "-x", "20", "-g", "d"]

[parameter.x]
values = ["a", "b"]

[parameter.y]
values = ["10", "20"]
        \end{verbatim}

        Example of wrong config:

        \begin{verbatim}
[input.testrun1]
arguments = ["subparam|x.1", "param|x", "subparam|x.2"]

[parameter.x.sub.1]
values = ["a", "b"]

[parameter.x.sub.2]
values = ["c", "d"]

[parameter.x]
values = ["10", "20"]
        \end{verbatim}

        This is not correct and \Prog{gourd} will throw an error!

  \section{LABELS}
  \subsection{Labels}
  When running \Prog{gourd} \Arg{status}, by default the statuses only display information
  about Slurm scheduling of the run or an exit code.
  In the case that a job execution can succeed (exit code 0) but the run should still
  be considered a failure, the user can add a custom label to the run, derived from
  the output of the run's `afterscript'.
  Labels can be created in the configuration file as names with a regular expression,
  where if the regex is matched in the afterscript's output, the label is assigned to the run.
  Syntax is as follows:

  \File{gourd.toml}
  \begin{verbatim}
    [label.<label_name>]
    regex = <regex>
    priority = <int>
    rerun_by_default = <true/false>
  \end{verbatim}

  \begin{itemize}
      \item \File{<label\_name>}: The name of the label as is to be displayed by \Prog{gourd} \Arg{status}.
      \item \File{<regex>}: The regular expression to match against the afterscript's output.
      If a match is found, the label is assigned to the run.
      \item \File{<int>}: The priority of the label.
      Higher numbers have higher priority.
      If multiple labels match the afterscript's output, the label with the highest priority is assigned to the run.
      Multiple labels with the same priority are assigned to the run cause undefined behaviour.
      %(I think they are ordered alphabetically by name's hash).
      \item \File{<true/false>}: Including this is optional, with a default value of \File{true}.
      If the field is not present or present with value \File{true}, the run will be rerun by default if
      this label is assigned to it; i.e., this will be considered a failed run.
      If false then \Prog{gourd} \Arg{status} will display the run as successful, and \Prog{gourd} \Arg{rerun} will ignore it.
  \end{itemize}

  Labels are assigned based on priority.
  For example if the configuration file looks like:
  \begin{verbatim}
    [label.label1]
    regex = "Success"
    priority = 1
    rerun_by_default = false

    [label.label2]
    regex = "RuntimeException"
    priority = 2
  \end{verbatim}

  and the afterscript output looks like:

  \begin{verbatim}
    Starting afterscript...
    Success! The output was correct.
    Verifying something else...
    RuntimeException thrown while parsing
  \end{verbatim}

  then by principle of priorities, the run will be assigned \File{label2} even though both regex match.

  \section{Remote resource fetching}
    In order to prevent having to manually transfer large files, input files or (precompiled) program binaries can be fetched from a URL.

    Suppose you are hosting a large text file, or want to download a binary from CI artefacts:
    
    \texttt{https://example.com/resources/some\_input.txt}
   
     \texttt{https://example.com/programs/program.exe}

    You can pass this as an input by:
\begin{verbatim}
[input.some_input]
input = “https://example.com/resources/some_input.txt > ./path/to/local/file.txt”
arguments = [“any”, “input”, “arguments”]
\end{verbatim}

    This will download the file at the URL and save its contents in the provided path.
    The contents of this will then be passed as input to all programs.

    Note that the “>” character needs to be escaped in URLs as \texttt{\%3E} since it is used as a delimiter

    Similarly for programs,
\begin{verbatim}
[program.some_example]
binary = “https://example.com/programs/program.exe > ./path/to/local/program.exe”
arguments = [“any”, “program”, “arguments”]
\end{verbatim}
    Note: programs need to be precompiled for the same architecture as the system that is running gourd.

  \section{SEE ALSO}
      \Prog{gourd(1)}

  \section{AUTHORS}
    Lukáš Chládek <\Email{l@chla.cz}>\\[0.1cm]\MANbr
    Rūta Giedrytė <\Email{r.giedryte@student.tudelft.nl}>\\[0.1cm]\MANbr
    Ανδρέας Τσατσάνης <\Email{a.tsatsanis@student.tudelft.nl}>\\[0.1cm]\MANbr
    Mikołaj Gazeel <\Email{m.j.gazeel@student.tudelft.nl}>\\[0.1cm]\MANbr
    Jan Piotrowski <\Email{me@jan.wf}>
%@% IF LATEX %@%
\end{adjustwidth}
%@% END-IF %@%

\end{document}
