\documentclass[a4paper,english]{article}
\usepackage{a4wide}
\usepackage{babel}
\usepackage{verbatim}

\usepackage{changepage}

\usepackage[bookmarksopen,bookmarksnumbered]{hyperref}

\usepackage[fancyhdr]{latex2man}

\newcommand{\thecmd}{gourd.toml}
\newcommand{\thecommand}{GOURD.TOML}
\newcommand{\mansection}{1}
\newcommand{\mansectionname}{File Formats Manual}
\newcommand{\mandate}{19 AUGUST 2024}
\setDate{19 AUGUST 2024}
\setVersionWord{Version:}
\setVersion{1.1.0}

\fancyhead[L, R]{\textit{\thecommand}(\mansection)}
\fancyhead[C]{\textsc{\mansectionname}}
\fancyfoot[C]{\mandate}
\fancyfoot[R]{\thepage}
\renewcommand{\headrulewidth}{0pt}

\renewcommand{\Prog}[1]{\textbf{#1}}                      % Program name

\renewenvironment{abstract}{\noindent\bfseries NAME\normalfont\vspace{0.2cm}}{}

\renewenvironment{Name}[5]{
\title{#5}
\author{#3}
\date{\@LM@Date\\{\small Version \@LM@Version}}
\begin{abstract}
}{
\end{abstract}
\vspace{-0.8cm}
}

\renewenvironment{Description}[1][]{
\begin{list}{}{
 \ifthenelse{\equal{#1}{}}{
   % optional argument not given
     \labelwidth\z@ \itemindent-\leftmargin
     \let\makelabel\descriptionlabel
     \renewcommand{\makelabel}[1]{\hspace\labelsep\normalfont\bfseries##1}
  }{
   % optional argument given
   \settowidth{\labelwidth}{\normalfont\bfseries#1}
   \setlength{\leftmargin}{\labelwidth}
   \addtolength{\leftmargin}{\labelsep}
   \renewcommand{\makelabel}[1]{\normalfont\bfseries##1\hfil}
  }}
}{
\end{list}
}

\usepackage{titlesec}
\titleformat{\section}
  {\normalfont\normalsize\bfseries}{\thesection}{0.5em}{}

\titleformat{\subsection}
  {\normalfont\normalsize\bfseries}{\thesubsection}{0.5em}{}

\setlength{\parindent}{0pt}
\setlength{\parskip}{6pt}

% Improve the \section command if in a LaTeX environment.
%@% IF LATEX %@%
\let\oldsection\section
\renewcommand{\section}[1]{%
\end{adjustwidth}%
\oldsection*{#1}%
\begin{adjustwidth}{18pt}{0pt}%
}
\let\oldsubsection\subsection
\renewcommand{\subsection}[1]{%
\oldsubsection*{#1}%
}
\let\oldsubsubsection\subsubsection
\renewcommand{\subsubsection}[1]{%
\oldsubsubsection*{#1}%
}

\newcommand{\ddash}{-{}-}
%@% END-IF %@%


\usepackage{mathspec}
\setmainfont[Mapping=tex-text, FakeBold=1]{Linux Libertine O}
\setmathfont(Digits,Greek,Latin)[Numbers=OldStyle, FakeBold=1]{Linux Libertine O}

\begin{document}
  \pagestyle{fancy}

  \begin{Name}{5}{gourd.toml}{Test}{File Formats Manual}{gourd.toml}
%@% IF LATEX %@%
\begin{adjustwidth}{18pt}{0pt}
%@% END-IF %@%

      \Prog{gourd.toml} \- An experimental setup file for the Gourd experiment scheduler.

%@% IF LATEX %@%
\end{adjustwidth}
%@% END-IF %@%
  \end{Name}

%@% IF LATEX %@%
\begin{adjustwidth}{18pt}{0pt}
%@% END-IF %@%

  \section{DESCRIPTION}

      \Prog{gourd.toml} is a declarative configuration file that,
      when provided to \Prog{gourd(1)}, can be used to create a new \emph{experiment}.
      The rules for typesetting the file follow the standard TOML format.

      By default, the filename is \File{gourd.toml}.

  \section{PREFACE}

      Values ending with `\texttt{?}' can be omitted, the default value
      for them is mentioned in their description.

      \subsection{VALUE TYPES}
          The configuration uses the following value types.

          \begin{Description}[Types]\setlength{\itemsep}{0cm}
              \item[string] A string of characters surrounded by "".
              \item[path] A file path surrounded by "".
              \item[fetched\_path] Information about this is in the \textbf{RESOURCE FETCHING} section.
              \item[boolean] Either \emph{true} or \emph{false}.
              \item[number] Zero or a positive number.
              \item[regex] A regular expression.
              \item[duration] An human readable amount of time, for example: "2d 4h".
              \item[list of T] A list of values of type T surrounded by [].
          \end{Description}

  \section{GLOBAL CONTEXT}

    These should be put at the beginning of the file before defining any sections.

    \begin{Description}[Options]\setlength{\itemsep}{0cm}
        \item[\Opt{output\_path} = path]
        This path specifies where to store the stdout, stderr, and afterscript
        outputs for \emph{programs}.

        \item[\Opt{metrics\_path} = path]
        Where to store the metrics for \Prog{gourd(1)} \Arg{status}.
        These metrics contain information such as: Wall clock time,
        User time, System time, The amount of context switches etc.

        \item[\Opt{experiments\_folder} = path]
        Where to store state of previously ran experiments.

        Essentially this folder specifies where \Prog{gourd} will store
        all of its information about experiments. If this folder is removed
        \Prog{gourd} looses all information about past experiments.

        \item[\Opt{wrapper?} = path]
        Defines the path to the \File{gourd\_wrapper} binary. \\ \\
        The default value is \emph{gourd\_wrapper}. (That is \Prog{gourd}
        will look for it in the \texttt{\$PATH}) \\ \\
        If you installed \Prog{gourd} correctly this values should not be
        changed.

        \item[\Opt{input\_schema?} = path]
        Defines the path to a optional machine generated input schema. \\ \\
        For more information about this continue to the \textbf{INPUT SCHEMA}
        section. \\ \\
        The default values is no input schema.

        \item[\Opt{warn\_on\_label\_overlap?} = bool]
        Information about this can be found in the \textbf{LABELS}
        section. \\ \\
        The default value is \emph{false}.
    \end{Description}

    \section{SLURM}

      The configuration contains some Slurm specific options namely:

      \subsection{SLURM CONFIGURATION}

          This section is optional for running locally, but required
          for running on Slurm.

          A Slurm configuration starts with the header \Arg{[slurm]}
          and can contain the following options:

          \begin{Description}[Options]\setlength{\itemsep}{0cm}
              \item[\Opt{experiment\_name} = string]
                The name under which runs for this experiment will be scheduled
                on Slurm.
              \item[\Opt{output\_folder} = path]
                The folder where the raw slurm outputs will be stored. This can be useful
                for debugging.
              \item[\Opt{partition} = string]
                The partition on which this should be ran on the supercomputer.
                Running \Prog{gourd} \Arg{run} \Arg{slurm} with an invalid partition
                will display all of the valid partitions.
              \item[\Opt{array\_size\_limit?} = number]
                This specifies the limit of runs that can be put in one Slurm batch.
                \Prog{gourd} will work to split the workload such that this limit
                is never exceeded. \\ \\
                By default this is auto-detected.
              \item[\Opt{max\_submit?} = number]
                This specifies the limits of possible batches of runs.
                Once again \Prog{gourd} will work to never cross this limit. \\ \\
                By default this is auto-detected.
              \item[\Opt{account} = string]
                Which account to use for running jobs on Slurm.
                For example one account available on DelftBlue is "Education-EEMCS-MSc-CS".
              \item[\Opt{additional\_args?} = list of string]
                Custom arguments for Slurm. \\ \\
                By default there are no additional arguments.
          \end{Description}

          \subsubsection{Example}
              An example Slurm Configuration:

              \begin{verbatim}
[slurm]
experiment_name = "test experiment"
output_folder = "./slurmout/"
partition = "compute"
account = "Education-EEMCS-MSc-CS"
              \end{verbatim}

      \subsection{RESOURCE LIMITS}

          To run on Slurm one must also specify resource limits.
          The available limits are:

          \begin{Description}[Options]\setlength{\itemsep}{0cm}
              \item[\Opt{time\_limit} = duration]
                The global time limit for all program-input pairs.
              \item[\Opt{cpu\_limit} = number]
                The global cpu limit for all program-input pairs.
              \item[\Opt{mem\_per\_cpu} = number]
                The global memory limit per one cpu for all program-input pairs. \\ \\
                This number is specified in megabytes.
          \end{Description}


          \subsubsection{Example}
              An example Resource Limits section:

              \begin{verbatim}
[resource_limits]
time_limit = "5min"
cpus = 1
mem_per_cpu = 512
              \end{verbatim}

          \subsubsection{Postprocessing Resource Limit}

              If \Prog{gourd} is using postprocessing jobs a separate section
              (with the same fields) called \\
              \texttt{[postprocess\_resource\_limits]} must be specified.
              It holds the same fields.

    \section{PROGRAMS}

        Multiple programs can be specified.
        A program represents a compiled algorithm and is a combination of a binary file and parameters.
        Each program begins with \Arg{[programs.program-name]}, where \Arg{program-name} can be any unique name.

        \begin{Description}[Options]\setlength{\itemsep}{0cm}
            \item[\Opt{binary} = path]
              Path to the program executable.
            \item[\Opt{fetch} = fetched\_path]
              URL to the program executable.
            \item[\Opt{arguments?} = list of string]
              Arguments to be passed to the executable. \\ \\
              By default an empty list.
            \item[\Opt{afterscript?} = path]
              See the \textbf{AFTERSCRIPTS} section for more information. \\ \\
              By default there is no afterscript.
            \item[\Opt{next?} = list of string]
              See the \textbf{POSTPROCESSING} section for more information. \\ \\
              By default there is no postprocessing.
            \item[\Opt{resource\_limits?}]
              As defined in the \textbf{RESOURCE LIMITS} section. \\ \\
              These essentially override the global resource limits for
              this program. \\ \\
              By default, use the global resource limits.
        \end{Description}

        Only one of \Opt{binary}, \Opt{fetch} must be specified.

        \subsection{EXAMPLE}

            Assume that there is a script called \texttt{test.sh} in the current directory.
            We can specify a program that runs this script with the argument \Opt{\ddash test}
            like so:

            \begin{verbatim}
[program.some_name_for_this_program]
binary = "./test.sh"
arguments = ["--test"]
            \end{verbatim}

    \section{INPUTS}

        A \Prog{gourd(1)} experiment consists of a cross-product mapping between programs
        and inputs.
        The experiment created from a \File{gourd.toml} file runs every combination
        of program and input in the file.

        Multiple inputs can be specified.
        Each input begins with \Arg{[inputs.input-name]} where \Arg{input-name} can be any unique name.
        \textbf{The string} \Arg{\_i\_} \textbf{is reserved and cannot be used.}
        Each input contains the following keys:

        \begin{Description}[Options]\setlength{\itemsep}{0cm}
            \item[\Opt{file?} = path]
              Path to a file, the contents of which are passed to the program as standard input. \\ \\
              By default, no standard input is provided.
            \item[\Opt{fetch?} = fetched\_path]
              URL to a file, the contents of which are passed to the program as standard input.
            \item[\Opt{glob?} = fetched\_path]
              A glob expression of multiple files, the contents of which are passed to the program
              as standard input.
            \item[\Opt{arguments?} = list of string]
              Additional command-line arguments to be passed to the program.
              The input arguments are appended to the programs arguments. \\ \\
              By default, there are no additional arguments.
        \end{Description}

        Only one of \Opt{file}, \Opt{fetch}, \Opt{glob} can be specified.

        \subsection{EXAMPLE}
            A valid input would be for example:

            \begin{verbatim}
[input.some_name_for_this_input]
file = "./test.txt"
arguments = ["--a", "--b"]
            \end{verbatim}
            This applied to program `\texttt{program}' would be equivalent to:

            \begin{verbatim}
program [program args] --a --b < ./test.txt
            \end{verbatim}


      \subsection{GLOBS}

        Globs can be applied to arguments of inputs and conveniently reference multiple files.

        If an argument starts with \Arg{path|}, it will be treated as a glob.
        The input will be instantiated for every match of the provided glob.

        \subsubsection{Example}

            \begin{verbatim}
[inputs.testrun1]
arguments = ["-f1", "path|./inputs/*.in", "-f2", "path|./input2/*.in"]
            \end{verbatim}

            Given that the current directory contains the files
            \File{input/1.in}, \File{input/2.in}, \File{input2/test.in}, the glob
            expands to the following experiment inputs:

            \begin{verbatim}
[inputs.testrun1_glob_0]
arguments = ["-f1", "./inputs/1.in", "-f2", "./input2/test.in"]

[inputs.testrun1_glob_1]
arguments = ["-f1", "./inputs/2.in", "-f2", "./input2/test.in"]
            \end{verbatim}

      \subsection{PARAMETERS}

          Parameters can be applied to arguments to conveniently perform experiments with grid search
          (a Cartesian product between all parameter values).

          If an argument starts with \Arg{param|some-parameter-name}, it will be treated as a parameter.
          For each value of that parameter the new input will be created with that value inserted into the argument
          into that argument place.

          Values of a parameter are specified in \Arg{[parameter.name]} using \Arg{values = list of string}

          This results in cross product between all parameters.

          \subsubsection{Example}

              \begin{verbatim}
[inputs.testrun1]
arguments = ["-f", "param|x", "-x", "param|y"]

[parameters.x]
values = ["a", "b"]

[parameters.y]
values = ["10", "20"]
        \end{verbatim}
                It will be transformed into following inputs:

        \begin{verbatim}
[inputs.testrun1_x_0_y_0]
arguments = ["-f", "a", "-x", "10"]

[inputs.testrun1_x_0_y_1]
arguments = ["-f", "a", "-x", "20"]

[inputs.testrun1_x_1_y_0]
arguments = ["-f", "b", "-x", "10"]

[inputs.testrun1_x_1_y_1]
arguments = ["-f", "b", "-x", "20"]
            \end{verbatim}

    \subsection{SUBPARAMETERS}

        Subparameters are used when there is a need for 1-1 relation between two parameters.
        There is no cross product between subparameters of the same parameter.

        Subparameters are specified in inputs similarly to parameters with the difference of doing
        \Arg{subparam|parameter-name.some-subparameter-name}.

        Values of a subparameter are specified in \Arg{[parameter.name.sub.subparameter-name]} using \Arg{values = []}

        Note! Parameters can have either values or subparameters with values. Never both.

        \subsubsection{Example}
            This example:

        \begin{verbatim}
[input.testrun1]
arguments = ["-f", "subparam|x.1", "-x", "param|y", "-g", "subparam|x.2"]

[parameter.x.sub.1]
values = ["a", "b"]

[parameter.x.sub.2]
values = ["c", "d"]

[parameter.y]
values = ["10", "20"]
        \end{verbatim}
            Will be transformed into following inputs:

        \begin{verbatim}
[input.testrun1_x-0_y-0]
arguments = ["-f" "a", "-x", "10", "-g", "c"]

[input.testrun1_x-0_y-1]
arguments = ["-f", "a", "-x", "20", "-g", "c"]

[input.testrun1_x-1_y-0]
arguments = ["-f", "b", "-x", "10", "-g", "d"]

[input.testrun1_x-1_y-1]
arguments = ["-f", "b", "-x", "20", "-g", "d"]

[parameter.x]
values = ["a", "b"]

[parameter.y]
values = ["10", "20"]
        \end{verbatim}
            Where as this example:

        \begin{verbatim}
[input.testrun1]
arguments = ["param|x"]

[parameter.x.sub.1]
values = ["a", "b"]

[parameter.x.sub.2]
values = ["c", "d"]

[parameter.x]
values = ["10", "20"]
        \end{verbatim}

        Is not correct and \Prog{gourd} will throw an error!



    \section{POSTPROCESSING}

        Postprocessing jobs are jobs that run after another job and
        can transform its input without influencing the original jobs input.

        Postprocessing programs are just normal programs but they a different
        program references them in their \texttt{next} field.

        Postprocessing programs are ran in the same directory as the original
        job, and get the originals job \texttt{stdout} as their \texttt{stdin}.

        \subsection{EXAMPLE}

            \begin{verbatim}
[program.test_program]
binary = "./algorithm"
arguments = []
next = ["example_name"]

[program.example_name]
binary = "./verifier"
arguments = []
            \end{verbatim}

    \section{AFTERSCRIPTS}

        Afterscripts are postprocessing but one that does not constitute
        a full Slurm job.

        These are ran when \Prog{gourd} \Arg{status} is invoked, and their results are
        cached.

        The afterscript is assumed to:

        \begin{itemize}
            \item Be executable.
            \item Will receive the path to the jobs output as the first CLI parameter.
            \item Write its output to the path received as the second CLI parameter.
        \end{itemize}

        \subsection{EXAMPLE}

            \begin{verbatim}
[program.test_program]
binary = "./algorithm"
arguments = []
afterscript = "./script.sh"
            \end{verbatim}

            After running the job the after script will be called as:

            \begin{verbatim}
script.sh path/to/job/stdout path/to/afterscript/output
            \end{verbatim}

            And for example if:

            in \textbf{script.sh}:
            \begin{verbatim}
cat $1 > $2
            \end{verbatim}

            The afterscripts output will be the jobs output (ie. No postprocessing happened).
            But these scripts may be more complex if the use case requires it.

    \section{LABELS}

      When running \Prog{gourd} \Arg{status}, by default the statuses only display information
      about Slurm scheduling of the run or an exit code.

      In the case that a job execution can succeed (exit code 0) but the run should still
      be considered a failure, the user can add a custom label to the run, derived from
      the output of the run's `afterscript'.

      Labels can be created in the configuration file as names with a regular expression,
      where if the regex is matched in the afterscript's output, the label is assigned to the run.

      These are specified as \Arg{[label.label-name]} and the fields available are:

      \begin{Description}[Options]\setlength{\itemsep}{0cm}
          \item[\Opt{regex} = regex]
          A regular expression that the afterscripts output will be matched to,
          if the output matches the expression this label will be assigned.

          \item[\Opt{priority} = number]
          In the case that more than one label matches a run the \textbf{highest}
          priority label will be assigned.

          \item[\Opt{rerun\_by\_default?} = boolean]
          If true makes this label essentially mean `failure', in the sense that
          \Prog{gourd} will treat a run with this label as a failure even if the
          run itself succeeded. \\ \\
          By default \emph{false}.
      \end{Description}

      \subsection{EXAMPLE}

          \begin{verbatim}
[label.label_name]
# matches any output
regex = ".*"
priority = 1
rerun_by_default = true
          \end{verbatim}
          Labels are assigned based on priority.
          For example if the configuration file looks like:

          \begin{verbatim}
[label.label1]
regex = "Success"
priority = 1
rerun_by_default = false

[label.label2]
regex = "RuntimeException"
priority = 2
          \end{verbatim}
          and the afterscript output looks like:

          \begin{verbatim}
Starting afterscript...
Success! The output was correct.
Verifying something else...
RuntimeException thrown while parsing
          \end{verbatim}
          then by principle of priorities, the run will be assigned
          \emph{label2} even though both regexes match.

    \section{REMOTE RESOURCE FETCHING}

        In order to prevent having to manually transfer large files, input files
        or (precompiled) program binaries can be fetched from a URL.

        Suppose you are hosting a large text file, or want to download a binary from CI artefacts:

        \texttt{https://test.com/input.txt}

        \texttt{https://test.com/program.exe}

        Any config field which accepts \texttt{fetched\_path} can accept remote resources.

        The syntax for fetched resources is "\texttt{remote\_path | local\_path}"
        (whitespace insensitive).

        \subsection{EXAMPLE}

            Consider this fetched input as an example:

            \begin{verbatim}
[input.some_input]
fetch = “https://test.com/input.txt | ./path/to/store/the/file.txt”
arguments = [“any”, “input”, “arguments”]
            \end{verbatim}
            This will download the file at \texttt{test.com/input.txt} and save it
            contents in the provided path.
            The contents of this will then be passed as input to all programs.

            Note that the “|” character needs to be escaped in URLs as
            \texttt{\%7C} since it is used as a delimiter

            Similarly for programs:

            \begin{verbatim}
[program.some_example]
fetch = “https://test.com/program.exe | ./path/to/store/the/program.exe”
arguments = [“any”, “program”, “arguments”]
            \end{verbatim}

        \subsection{CACHING}

            These resources will be downloaded and saved at the paths, but they will
            not be redownloaded again as long as these cached files exist.

            It may be beneficial to create a folder and store all downloaded resources
            inside, then when there is a need for cleaning the cache this amounts
            to deleting the folder

  \section{SEE ALSO}
      \Prog{gourd(1)} \Prog{gourd-tutorial(7)}

  \section{AUTHORS}
    Lukáš Chládek <\Email{l@chla.cz}>\\[0.1cm]\MANbr
    Rūta Giedrytė <\Email{r.giedryte@student.tudelft.nl}>\\[0.1cm]\MANbr
    Ανδρέας Τσατσάνης <\Email{a.tsatsanis@student.tudelft.nl}>\\[0.1cm]\MANbr
    Mikołaj Gazeel <\Email{m.j.gazeel@student.tudelft.nl}>\\[0.1cm]\MANbr
    Jan Piotrowski <\Email{me@jan.wf}>
%@% IF LATEX %@%
\end{adjustwidth}
%@% END-IF %@%

\end{document}
