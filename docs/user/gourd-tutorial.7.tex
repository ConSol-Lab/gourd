\documentclass[a4paper,english]{article}
\usepackage{a4wide}
\usepackage{babel}
\usepackage{verbatim}

\usepackage{changepage}

\usepackage[bookmarksopen,bookmarksnumbered]{hyperref}

\usepackage[fancyhdr]{latex2man}

\usepackage{xspace}



\newcommand{\thecmd}{gourd}
\newcommand{\thecommand}{GOURD-TUTORIAL}
\newcommand{\mansection}{7}
\newcommand{\mansectionname}{DelftBlue Tools Manual}
\newcommand{\mandate}{5 MAY 2024}
\setDate{5 MAY 2024}
\setVersionWord{Version:}
\setVersion{1.29}


\fancyhead[L, R]{\textit{\thecommand}(\mansection)}
\fancyhead[C]{\textsc{\mansectionname}}
\fancyfoot[C]{\mandate}
\fancyfoot[R]{\thepage}
\renewcommand{\headrulewidth}{0pt}

\renewcommand{\Prog}[1]{\textbf{#1}}                      % Program name

\renewenvironment{abstract}{\noindent\bfseries NAME\normalfont\vspace{0.2cm}}{}

\renewenvironment{Name}[5]{
\title{#5}
\author{#3}
\date{\@LM@Date\\{\small Version \@LM@Version}}
\begin{abstract}
}{
\end{abstract}
\vspace{-0.8cm}
}

\renewenvironment{Description}[1][]{
\begin{list}{}{
 \ifthenelse{\equal{#1}{}}{
   % optional argument not given
     \labelwidth\z@ \itemindent-\leftmargin
     \let\makelabel\descriptionlabel
     \renewcommand{\makelabel}[1]{\hspace\labelsep\normalfont\bfseries##1}
  }{
   % optional argument given
   \settowidth{\labelwidth}{\normalfont\bfseries#1}
   \setlength{\leftmargin}{\labelwidth}
   \addtolength{\leftmargin}{\labelsep}
   \renewcommand{\makelabel}[1]{\normalfont\bfseries##1\hfil}
  }}
}{
\end{list}
}

\usepackage{titlesec}
\titleformat{\section}
  {\normalfont\normalsize\bfseries}{\thesection}{0.5em}{}

\titleformat{\subsection}
  {\normalfont\normalsize\bfseries}{\thesubsection}{0.5em}{}

\setlength{\parindent}{0pt}
\setlength{\parskip}{6pt}

% Improve the \section command if in a LaTeX environment.
%@% IF LATEX %@%
\let\oldsection\section
\renewcommand{\section}[1]{%
\end{adjustwidth}%
\oldsection*{#1}%
\begin{adjustwidth}{18pt}{0pt}%
}
\let\oldsubsection\subsection
\renewcommand{\subsection}[1]{%
\oldsubsection*{#1}%
}
\let\oldsubsubsection\subsubsection
\renewcommand{\subsubsection}[1]{%
\oldsubsubsection*{#1}%
}

\newcommand{\ddash}{-{}-}
%@% END-IF %@%


\usepackage{mathspec}
\setmainfont[Mapping=tex-text, FakeBold=1]{Linux Libertine O}
\setmathfont(Digits,Greek,Latin)[Numbers=OldStyle, FakeBold=1]{Linux Libertine O}

\begin{document}
    \pagestyle{fancy}

    \begin{Name}{7}{gourd-tutorial}{gourd-tutorial}{DelftBlue Tools Manual}{Gourd Tutorial}
%@% IF LATEX %@%
\begin{adjustwidth}{18pt}{0pt}
%@% END-IF %@%

        \Prog{gourd-tutorial} - A step-by-step walkthrough for the Gourd experiment scheduler.

%@% IF LATEX %@%
\end{adjustwidth}
%@% END-IF %@%
    \end{Name}

%@% IF LATEX %@%
\begin{adjustwidth}{18pt}{0pt}
%@% END-IF %@%

    \section{INTRODUCTION}

    Welcome to \Prog{gourd-tutorial}!

    If you haven't been introduced yet, \Prog{gourd} is an application that makes it easy
    to set up experiments on a supercomputer.
    By experiment, we mean a large-scale comparative evaluation of one or more
    \emph{algorithms} (runnable programs) that each run on a set of \emph{inputs} and are
    subsequently timed and profiled.

    While this tool offers a lot of versatility, this set of runnable examples will show
    that \Prog{gourd} experiments only take a minute to set up.

    \section{INSTALLATION AND REFERENCE}

    This tutorial is designed to be interactive, so be sure to have a working copy of
    \Prog{gourd} installed on your computer.
    You can verify this by typing \Prog{gourd}~\Arg{version} in your terminal.
    For installation instructions, refer to the \File{README.md} file in the source
    repository.

    When installed, you will also have access to the user manuals.
    For Linux, macOS, and the like, type \Prog{man}~\Prog{gourd-tutorial} to see this
    tutorial or \Prog{gourd} and \Prog{gourd.toml} for complete documentation.

    \section{INTERACTING WITH GOURD}

    Gourd is a command-line application that keeps life easy.
    You take actions by typing \Prog{gourd} followed by a command in your terminal;
    a complete list is in the manual.

    For example, type:
    \Prog{gourd}~\Arg{init}~\Arg{--example}~\Arg{a-simple-experiment}~\Arg{myexample}

    The \Prog{gourd}~\Arg{init} command will set up the \File{myexample} folder to match
    the example below!
    Make sure to use \Prog{cd} to enter the folder if you're following along.


    \section{A SIMPLE EXPERIMENT}

    In \Prog{gourd}, each experimental setup is defined using a \File{gourd.toml} file.
    To make one, let's open \File{gourd.toml} in an editor and start by defining a pair
    of \textbf{programs}.

% -IF LINUX/OSX
    \begin{verbatim}
…  +-------------------------------------------------------------------+
1  | [programs.my-program]                                             |
2  | binary = "/bin/sleep"                                             |
3  |                                                                   |
4  | [programs.my-other-program]                                       |
5  | binary = "/bin/bash"                                              |
6  | args = "./my_other_program.sh"                                    |
7  |                                                                   |
…  /_`_`_`_`_`_`_`_`_`_`_`_`_`_`_`_`_`_`_`_`_`_`_`_`_`_`_`_`_`_`_`_`_`_/
…  \___________________________________________________________________\
    \end{verbatim}

    In this file, we have defined two programs with unique names:
    \begin{Description}[programs]\setlength{\itemsep}{0cm}
    \item[my-program:] the \File{sleep} binary, which waits a set amount of seconds,
    \item[my-other-program:] a \File{bash} script that returns whatever input it received.
    \end{Description}

% -ENDIF LINUX/OSX



% -IF WINDOWS
%    \begin{verbatim}
%
%   +-------------------------------------------------------------------+
%1  | [programs.MyProgram]                                              |
%2  | binary = "./myprogram.bat"                                        |
%3  |                                                                   |
%…  /_`_`_`_`_`_`_`_`_`_`_`_`_`_`_`_`_`_`_`_`_`_`_`_`_`_`_`_`_`_`_`_`_`_/
%   \___________________________________________________________________\
%
%    \end{verbatim}
% -ENDIF WINDOWS

    That's great, but now we need inputs: test cases that we actually run the
    programs with.
    Let's add those.

    \begin{verbatim}
…   ____________________________________________________________________
…  /_ _ _ _ _ _ _ _ _ _ _ _ _ _ _ _ _ _ _ _ _ _ _ _ _ _ _ _ _ _ _ _ _ _/
…  \                                                                   \
…  |` ` ` ` ` ` ` ` ` ` ` ` ` ` ` ` ` ` ` ` ` ` ` ` ` ` ` ` ` ` ` ` ` `|
8  | [inputs.one]                                                      |
9  | input = "./input-one"                                             |
10 |                                                                   |
11 | [inputs.two]                                                      |
12 | args = ["2"]                                                      |
13 |                                                                   |
14 | [inputs.one-two]                                                  |
16 | input = "./input-one"                                             |
17 | args = ["2"]                                                      |
18 |                                                                   |
…  /_`_`_`_`_`_`_`_`_`_`_`_`_`_`_`_`_`_`_`_`_`_`_`_`_`_`_`_`_`_`_`_`_`_/
…  \___________________________________________________________________\
    \end{verbatim}

    What you see now in the file are the following uniquely named inputs:
    \begin{Description}[inputs]\setlength{\itemsep}{0cm}
    \item[one:] a path to an \File{./input-one} file containing the string \File{"1"},
    \item[two:] an input specified by directly passing an argument, \File{"2"},
    \item[abc-one:] an input specified by a combination of the two.
    \end{Description}

    The inputs above are combined with the programs in a \textbf{cross product}
    to create \emph{runs}.
    This means that each program and each input are combined together.
    Therefore, exactly one run in our new experiment that combines
    \File{my-other-program} and \File{one-two}, and it translates into the following
    program call:
    \begin{itemize}
    \item Run \File{/bin/bash} with \File{./my-other-program.sh~2} as arguments and
    the contents of \File{./input-one} as input.
    \end{itemize}

    \section{SEE ALSO}

    \Prog{gourd}(1)

    \Prog{gourd.toml}(5)

    \section{AUTHORS}
    Lukáš Chládek <\Email{l@chla.cz}>\\[0.1cm]\MANbr
    Rūta Giedrytė <\Email{r.giedryte@student.tudelft.nl}>\\[0.1cm]\MANbr
    Ανδρέας Τσατσάνης <\Email{a.tsatsanis@student.tudelft.nl}>\\[0.1cm]\MANbr
    Mikołaj Gazeel <\Email{m.j.gazeel@student.tudelft.nl}>\\[0.1cm]\MANbr
    Jan Piotrowski <\Email{j.p.piotrowski@student.tudelft.nl}>
%@% IF LATEX %@%
\end{adjustwidth}
%@% END-IF %@%

\end{document}
