\documentclass[a4paper,english]{article}
\usepackage{a4wide}
\usepackage{babel}
\usepackage{verbatim}

\usepackage{changepage}

\usepackage[bookmarksopen,bookmarksnumbered]{hyperref}

\usepackage[fancyhdr]{latex2man}

\usepackage{xspace}



\newcommand{\thecmd}{gourd}
\newcommand{\thecommand}{GOURD-TUTORIAL}
\newcommand{\mansection}{7}
\newcommand{\mansectionname}{DelftBlue Tools Manual}
\newcommand{\mandate}{20 JUNE 2024}
\setDate{20 JUNE 2024}
\setVersionWord{Version:}
\setVersion{1.0.0}


\fancyhead[L, R]{\textit{\thecommand}(\mansection)}
\fancyhead[C]{\textsc{\mansectionname}}
\fancyfoot[C]{\mandate}
\fancyfoot[R]{\thepage}
\renewcommand{\headrulewidth}{0pt}

\renewcommand{\Prog}[1]{\textbf{#1}}                      % Program name

\renewenvironment{abstract}{\noindent\bfseries NAME\normalfont\vspace{0.2cm}}{}

\renewenvironment{Name}[5]{
\title{#5}
\author{#3}
\date{\@LM@Date\\{\small Version \@LM@Version}}
\begin{abstract}
}{
\end{abstract}
\vspace{-0.8cm}
}

\renewenvironment{Description}[1][]{
\begin{list}{}{
 \ifthenelse{\equal{#1}{}}{
   % optional argument not given
     \labelwidth\z@ \itemindent-\leftmargin
     \let\makelabel\descriptionlabel
     \renewcommand{\makelabel}[1]{\hspace\labelsep\normalfont\bfseries##1}
  }{
   % optional argument given
   \settowidth{\labelwidth}{\normalfont\bfseries#1}
   \setlength{\leftmargin}{\labelwidth}
   \addtolength{\leftmargin}{\labelsep}
   \renewcommand{\makelabel}[1]{\normalfont\bfseries##1\hfil}
  }}
}{
\end{list}
}

\usepackage{titlesec}
\titleformat{\section}
  {\normalfont\normalsize\bfseries}{\thesection}{0.5em}{}

\titleformat{\subsection}
  {\normalfont\normalsize\bfseries}{\thesubsection}{0.5em}{}

\setlength{\parindent}{0pt}
\setlength{\parskip}{6pt}

% Improve the \section command if in a LaTeX environment.
%@% IF LATEX %@%
\let\oldsection\section
\renewcommand{\section}[1]{%
\end{adjustwidth}%
\oldsection*{#1}%
\begin{adjustwidth}{18pt}{0pt}%
}
\let\oldsubsection\subsection
\renewcommand{\subsection}[1]{%
\oldsubsection*{#1}%
}
\let\oldsubsubsection\subsubsection
\renewcommand{\subsubsection}[1]{%
\oldsubsubsection*{#1}%
}

\newcommand{\ddash}{-{}-}
%@% END-IF %@%


\usepackage{mathspec}
\setmainfont[Mapping=tex-text, FakeBold=1]{Linux Libertine O}
\setmathfont(Digits,Greek,Latin)[Numbers=OldStyle, FakeBold=1]{Linux Libertine O}

\begin{document}
    \pagestyle{fancy}

    \begin{Name}{7}{gourd-tutorial}{gourd-tutorial}{DelftBlue Tools Manual}{Gourd Tutorial}
%@% IF LATEX %@%
\begin{adjustwidth}{18pt}{0pt}
%@% END-IF %@%

        \Prog{gourd-tutorial} - A step-by-step walkthrough for the Gourd experiment scheduler.

%@% IF LATEX %@%
\end{adjustwidth}
%@% END-IF %@%
    \end{Name}

%@% IF LATEX %@%
\begin{adjustwidth}{18pt}{0pt}
%@% END-IF %@%

    \section{INTRODUCTION}

    Welcome to \Prog{gourd-tutorial}!

    If you haven't been introduced yet, \Prog{gourd(1)} is an application
    that makes it easy to set up experiments on a supercomputer.
    By experiment, we mean a large-scale comparative evaluation of one or
    more \emph{algorithms} (runnable programs) that each run on a set of
    \emph{inputs} and are subsequently timed and profiled.

    While this tool offers a lot of versatility, this set of runnable
    examples will show that \Prog{gourd} experiments only take a minute to
    set up.

    \section{INSTALLATION AND REFERENCE}

    This tutorial is designed to be interactive, so be sure to have a working copy of
    \Prog{gourd(1)} installed on your computer.
    You can verify this by typing \Prog{gourd}~\Arg{version} in a terminal.
    For installation instructions, refer to the \File{README.md} file in the
    source repository.

    When installed, you will also have access to the user manuals.
    For Linux, macOS, and the like, type \Prog{man}~\Prog{gourd-tutorial} to
    see this tutorial or \Prog{gourd} and \Prog{gourd.toml} for complete
    documentation.

    \section{INTERACTING WITH GOURD}

    Gourd is a command-line application that keeps life easy.
    You take actions by typing \Prog{gourd} followed by a command in your
    terminal; a complete list is in the manual.

    For example, type:
    \Prog{gourd}~\Arg{init}
    \Arg{--example}~\Arg{fibonacci-comparison}~\Arg{my\_fib}

    The \Prog{gourd}~\Arg{init} command will set up the \File{myexample} folder
    to match the example below!
    Furthermore, \Prog{gourd}~\Arg{init}~\Opt{--list-examples} will show what
    other examples are accessible to you.

    \section{FIBONACCI COMPARISON}

    Let's begin by designing a simple experiment.
    We will compare three versions of an algorithm that calculates Fibonacci
    numbers.

    First, let's define the experimental setup using a \File{gourd.toml}
    file. This file will specify the files, programs, and parameters of our
    setup in a reproducible way.


    The \Prog{gourd}~\Arg{init} command will set up the \File{myexample} folder to match
    the example below!
    Also, \Prog{gourd}~\Arg{init}~\Opt{--list-examples} will show what examples are accessible to you.


    \section{FIBONACCI COMPARISON}

    For our first experiment, we will compare three versions of an algorithm
    that calculates Fibonacci numbers.

    Open \File{gourd.toml} in an editor and type in the following lines:

    \begin{verbatim}

`                             ./gourd.toml
`  +-------------------------------------------------------------------+
1  | experiments_folder = "experiments"                                |
2  | metrics_path = "experiments"                                      |
3  | output_path = "experiments"                                       |
4  |                                                                   |
`  /_`_`_`_`_`_`_`_`_`_`_`_`_`_`_`_`_`_`_`_`_`_`_`_`_`_`_`_`_`_`_`_`_`_/
`  \___________________________________________________________________\
    \end{verbatim}

    In the TOML format, values (such as file paths) are in quotes (\verb|"|).
    You can also add comments using the \verb|#| character.

    The lines above set up the folder structure for our experiment's outputs.
    This particular setup puts everything in the same folder.

    Now, let's configure programs - the algorithms we are evaluating.

    \subsection{Defining programs}

    \begin{verbatim}
    ____________________________________________________________________
`  /_ _ _ _ _ _ _ _ _ _ _ _ _ _ _ _ _ _ _ _ _ _ _ _ _ _ _ _ _ _ _ _ _ _/
`  \                          ./gourd.toml                             \
`  |` ` ` ` ` ` ` ` ` ` ` ` ` ` ` ` ` ` ` ` ` ` ` ` ` ` ` ` ` ` ` ` ` `|
5  | [program.fibonacci]                                               |
6  | binary = "./fibonacci"                                            |
7  |                                                                   |
8  | [program.fast-fibonacci]                                          |
9  | binary = "./fibonacci-dynamic"                                    |
10 |                                                                   |
11 | [program.fastest-fibonacci]                                       |
12 | binary = "./fibonacci-dynamic"                                    |
13 | arguments = ["-f"]                                                |
14 |                                                                   |
`  /_`_`_`_`_`_`_`_`_`_`_`_`_`_`_`_`_`_`_`_`_`_`_`_`_`_`_`_`_`_`_`_`_`_/
`  \___________________________________________________________________\
    \end{verbatim}

    The lines above set up three uniquely named programs:
    \begin{Description}[programs]\setlength{\itemsep}{0cm}
    \item[fibonacci:] a slow Fibonacci number calculator.
    \item[fast-fibonacci] a faster version using Dynamic Programming.
    \item[fastest-fibonacci:] the same binary file as \Arg{fast-fibonacci} run
      with an additional command-line argument, \Opt{-f}, which should make it
      even faster!
    \end{Description}

    Each program links to a \Arg{binary} -- the executable file that runs our
    algorithm. In this case, our Fibonacci algorithms are compiled in Rust.
    If you are following this tutorial with
    \Prog{gourd}~\Arg{init}
    \Arg{--example}~\Arg{fibonacci-comparison},
    the folder contains both binaries:
    \File{fibonacci} and \File{fibonacci-dynamic}.

    In our evaluation, we are going to see how the three programs compare when
    running different test cases as inputs.
    Let's add inputs to our \File{gourd.toml}.

    \subsection{Defining inputs}

    \begin{verbatim}
    ____________________________________________________________________
`  /_ _ _ _ _ _ _ _ _ _ _ _ _ _ _ _ _ _ _ _ _ _ _ _ _ _ _ _ _ _ _ _ _ _/
`  \                          ./gourd.toml                             \
`  |` ` ` ` ` ` ` ` ` ` ` ` ` ` ` ` ` ` ` ` ` ` ` ` ` ` ` ` ` ` ` ` ` `|
15 | [input.test_2]                                                    |
16 | input = "./inputs/input_2"                                        |
17 |                                                                   |
18 | [input.test_8]                                                    |
19 | input = "./inputs/input_8"                                        |
20 |                                                                   |
21 | [input.test_35]                                                   |
22 | input = "./inputs/input_35"                                       |
23 |                                                                   |
24 | [input.bad_test]                                                  |
25 | input = "./inputs/input_bad"                                      |
26 |                                                                   |
`  /_`_`_`_`_`_`_`_`_`_`_`_`_`_`_`_`_`_`_`_`_`_`_`_`_`_`_`_`_`_`_`_`_`_/
`  \___________________________________________________________________\

    \end{verbatim}

    The lines above set up four uniquely named inputs.
    Each input refers to a file whose contents are fed into the program.

    In this example, inputs \File{test\_2}, \File{test\_8}, and \File{test\_35}
    link to files containing the numbers \File{2}, \File{8}, and \File{35}.
    These should make the Fibonacci algorithms output the 2nd, 8th, and 35th
    numbers of the Fibonacci sequence.
    The input named \File{bad\_test} contains \File{"some text"}, which isn't
    a valid number -- let's see how this will crash the programs.

    Inputs are combined with programs in a \textbf{cross product} to create
    \emph{runs}. Each program-input combination is exactly one \emph{run}.
    In this example, 3 programs * 4 inputs results in 12 \emph{runs}.

    \subsection{Running the evaluation}

    Our \File{gourd.toml} is complete - now it is time to run the evaluation
    using \Prog{gourd}~\Arg{run}.

    Typing \Prog{gourd}~\Arg{run} in a terminal will tell you that it has two
    subcommands:


    \begin{Description}[subcommands]\setlength{\itemsep}{0cm}
    \item[fibonacci:] a slow Fibonacci number calculator.
    \item[fast-fibonacci] a faster version using Dynamic Programming.
    \item[fastest-fibonacci:] the same binary file as \Arg{fast-fibonacci} run
      with an additional command-line argument, \Opt{-f}, which should make it
      even faster!
    \end{Description}


    \section{SEE ALSO}

    \Prog{gourd(1)}

    \Prog{gourd.toml(5)}

    \section{AUTHORS}
    Ανδρέας Τσατσάνης <\Email{a.tsatsanis@student.tudelft.nl}>\\[0.1cm]\MANbr
    Rūta Giedrytė <\Email{r.giedryte@student.tudelft.nl}>\\[0.1cm]\MANbr
    Mikołaj Gazeel <\Email{m.j.gazeel@student.tudelft.nl}>\\[0.1cm]\MANbr
    Jan Piotrowski <\Email{me@jan.wf}>
    Lukáš Chládek <\Email{l@chla.cz}>\\[0.1cm]\MANbr
%@% IF LATEX %@%
\end{adjustwidth}
%@% END-IF %@%

\end{document}
