\pagebreak


\section{Contributing}

The contribution guidelines for this project, and the decisions made about code style,
will be outlined in this section of the document.

\subsection{Guidelines}

The project is hosted on GitLab.
Changes to the ``main'' branch must be staged through a merge request approved by at least two maintainers.
The number of commits should be kept to a minimum to ensure clarity.

All commit messages must be lowercase.

No files can contain whitespace and the end of lines.

No files can contain newlines at the end.

All indentation should be spaces.

No merge commits, branches should always be rebased before a merge to \texttt{main}.

\subsection{GitHub Actions}

There are 3 pipelines in the project:
\subsubsection{Draft}
Build, test \& lint draft pull requests.
Quick and short pipeline with no artifacts.
This pipeline triggers on every push to a pull request with no reviewers requested, or when the \texttt{draft} label is explicitly assigned.
\subsubsection{Ready}
A longer pipeline than draft that includes documentation that is also published as an artifact.
Use this pipeline when the pull request is ready for review.
This pipeline triggers on every push to a pull request with reviewers requested, or when the \texttt{ready} label is explicitly assigned.
\subsubsection{Release}
Generates install scripts and documentation for the project.
Runs automatically on pushes to main, but can also be manually triggered by assigning the \texttt{release} label to a pull request.

\subsection{Releasing a new version}

Each merge request that increments the software version must do so in:

\begin{itemize}
  \item \texttt{Cargo.toml}
  \item The beginning of \texttt{docs/user/gourd.1.tex}
  \item The beginning of \texttt{docs/user/gourd.toml.5.tex}
  \item The beginning of \texttt{docs/user/gourd-tutorial.7.tex}
  \item In \texttt{docs/maintainer/preamble.tex}
\end{itemize}

\subsection{Documentation}

In addition to adding an entry to the version history, as detailed above, maintainers should also make sure that the
documentation remains thorough.
\begin{enumerate}
  \item The Maintainer PDF documentation (this document) contains a brief overview of the application's underlying design.
  \item The three ``manpages'' (in \verb|docs/user/|) contain detailed usage instructions and configuration documentation.
    The \texttt{gourd-tutorial} manpage is an exception: it is written in narrative form and should be accessible to new users.
  \item Inline documentation (Rustdoc) contains additional information necessary for the developer of a particular feature.
\end{enumerate}

\subsection{Code Style}

Code written in Rust should be formatted using `cargo fmt` and validated using `cargo clippy`.
All relevant code style rules are enforced by the GitLab CI and can be tested locally
by configuring the Git hooks:

The gitlab pipeline validates all of the changes against our rules, but to check these locally
we \textbf{strongly} recommend issuing the command below:

\begin{verbatim}
$ git config --local core.hooksPath .hooks/
\end{verbatim}

This will ensure that whenever you commit or push new changes they will be first checked against
a hook scripts which catch common mistakes that would have failed the pipeline check remotely.

Additionally, the following apply:
\begin{itemize}
  \item Beware when using \verb|[cfg]| statements, as they can cause `clippy' issues on different platforms.
  \item Enforce separation between crates (such as \gourd\ and \gourdwrap) as much as possible.
  \item Place tests in a separate file in a \verb|tests/| subdirectory of a module.
        Ensure that the filename corresponds to the file being tested.
\end{itemize}

\subsection{Testing}

The project goal is to be thoroughly tested with full-coverage systematic unit tests.
The GitLab environment analyzes coverage. This is displayed alongside each merge request.

\subsection{Integration tests}

\gourd\ has integration tests, they are contained in \texttt{src/integration} and
are full end-to-end command line tests.

There is an easy template for adding new integration tests which is stored in: \\
\texttt{src/integration/example.rs} please refer to that file if you add
new functionality to \gourd\ that should be tested with integration tests.
