\pagebreak

\section{Technologies}

\subsection{Rust}

Rust is the main programming language used for \gourd\ and its components.
The project uses Rust\footnote{See the MSRV for the version of Rust required}
edition 2021 for buidling.

\section{Structural Overview}\label{sec:architecture}

\subsection{The modules}

The \gourd\ project is divided into three core crates:
\gourd, \gourdlib, and \gourdwrap.

Their individual responsibilities are laid out in this section.

\subsubsection{\gourd\ -- Command-Line Application}

The \gourd\ command-line application is the core part of the project.
The source code is located in \verb|./src/gourd/|.

This compiles to the \gourd\ executable.

The \gourd\ binary uses the \gourdlib\ shared library to interface with

The responsibilities of the \gourd\ binary are to interact with the user,
schedule and run experiments (at a high level), and collect aggregate
status and metrics from the experiment's runs.

\subsubsection{\gourdwrap\ -- Wrapper Program}

The \gourdwrap\ is a binary that should not be invoked manually by the user.

The source code is located in \verb|./src/gourd-wrapper/|.

It is responsible for the low-level implementation of executing a run;
it takes care of combining runs of an individual binary program and input,
encapsulating the program in platform-native frameworks for collecting metrics.
This means that \gourdwrap\ is the actual executable scheduled once for each run.

\subsubsection{\gourdlib\ -- Wrapper Interface}

The \gourdlib\ crate contains all data shared between the application and the wrapper.
The source code is located in \verb|./src/gourd-lib/|.

This includes the `gourd.toml' configuration file (the formal definition of an experiment)
and an `experiment.lock' runtime data file.

The general pattern is that \gourd\ writes a `<experiment-number>.lock' TOML file when
an experiment is started, and \gourdwrap\ subsequently reads only the `lock' file and
on each execution to determine the path of the executable, the resource limits,et cetera

\subsection{Interactions}

Gourd currently contains 10 subcommands.
They can be divided into three types.

The first are commands that \emph{create a} experiment. These are:

\begin{itemize}
  \item \texttt{gourd run local}
  \item \texttt{gourd run slurm}
\end{itemize}

The second are commands that \emph{operate on} an experiment. These are:

\begin{itemize}
  \item \texttt{gourd status}
  \item \texttt{gourd rerun}
  \item \texttt{gourd continue}
  \item \texttt{gourd cancel}
  \item \texttt{gourd analyse}
  \item \texttt{gourd set-limits}
\end{itemize}


The third are miscellaneous commands that need neither an experiment
nor a config file. These are:

\begin{itemize}
  \item \texttt{gourd init}
  \item \texttt{gourd version}
\end{itemize}

\subsection{An overview of an experiments lifetime}

The first thing that a user will do is run either \texttt{gourd run local}
or \texttt{gourd run slurm}.

\gourd\ will look for the config file and \textbf{deserialize} it.
This is quite a big step and it is explained in detail in Section \ref{sec:serde}.

Then \gourd\ will prepare all of the runs of the experiment and create the
\textbf{Experiment} struct.

One can think of an \textbf{Experiment} as a \emph{actualization} of a \textbf{Config}.

This will create a experiment in the form of a \texttt{experiment-numer.lock}
file, this is a \texttt{toml} file which is the serialized version of the struct.

The first group of commands ends at this point. From now the second group can be used.

All of these conform to a similar structure:

\begin{enumerate}
  \item Deserialize the config to find the experiment folder.
  \item Read the experiment.
  \item Perform an operation (be it an iteraction with Slurm or the file system).
\end{enumerate}

\subsection{Interacting with Slurm}

This is one of the most a problematic areas of development.
Currently all interaction with Slurm is done via the commands:
\texttt{sbatch}, \texttt{scancel}, and \texttt{sacct}.

These commands do have a very stable interface and as such every time
\gourd\ uses them, it checks whether the version is compatible.

Moreover even though these commands have both a \texttt{-{}-yaml} and \texttt{-{}-json}
option, both of these have been found to be unrealiable (they block for very long and
sometimes may fail without reason).

We have therefore opted to use the \texttt{-p} or "parsable" output of these commands
for our interactions as this proved to be the most reliable.

Slurm also exposes a REST API for interaction as well as a dynamically loaded library,
these were not chosen due to previous attempts to integrate with them failing,
but we have the option of migration/adding more backends open.

The file which contains the \textbf{SlurmInteractor} trait is:
\texttt{src/gourd/slurm/mod.rs}.

To add a new Slurm backend the only needed change is to implement all of these
trait functions

\begin{figure}
  \begin{center}
    \begin{verbatim}
pub trait SlurmInteractor {
    fn get_version(&self) -> Result<[u64; 2]>;

    fn get_partitions(&self) -> Result<Vec<Vec<String>>>;

    fn schedule_chunk(
        &self,
        slurm_config: &SlurmConfig,
        chunk: &mut Chunk,
        chunk_id: usize,
        experiment: &mut Experiment,
        exp_path: &Path,
    ) -> Result<()>;

    fn is_version_supported(&self, v: [u64; 2]) -> bool;

    fn get_supported_versions(&self) -> String;

    fn get_accounting_data(&self, job_id: Vec<String>)
        -> Result<Vec<SacctOutput>>;

    fn get_scheduled_jobs(&self) -> Result<Vec<String>>;

    fn cancel_jobs(&self, batch_ids: Vec<String>) -> Result<()>;
}
    \end{verbatim}
  \end{center}
  \caption{The \textbf{SlurmInteractor} trait}
\end{figure}

Each of them are documented in the file itself and
an example of a implementation is the \textbf{SlurmCli} struct.

\subsection{The deserialization}\label{sec:serde}

One important part of the application is the deserialization of the
config file.

This process is spread out over all files in \texttt{src/gourd\_lib/config/}.

The config deserialization allows for things like expanding globs and
fetching, and other advanced features.

There are two flavours of deserialization for the \textbf{Config} struct,
`UserFacing', and `NonUserFacing'.

These are controlled by a \texttt{thread\_local!} variable.
While this may seem to be very bad idea at first, there is really no way
around it, for extensive rationale for this please see:
\texttt{src/gourd\_lib/config/maps.rs}

The `NonUserFacing' flavour resembles the default strict serde deserialization.
On the other hand the `UserFacing' flavour allows for extenstions such as:
\texttt{glob|}, \texttt{fetch|}, etc. (All documented in \texttt{gourd.toml.5.tex})

\subsection{On \texttt{gourd init}}\label{sec:builtin}

\texttt{gourd init} uses a very special building system for embedding its
built in examples into the binary. This build system is contained
in \texttt{src/resources/build\_builtin\_examples.rs} and it is gated with
a feature called \texttt{builtin-examples} (which is enabled by default).

Essentially, whenever \gourd\ is built all folders under
\texttt{src/resources/gourd\_init\_examples} are bundled as examples into the
application, the folder is bundled in whole with the exception of files ending in \texttt{.rs}.

Files ending in \texttt{.rs} will be compiled for the target architecture and included
as such.

To add a new example to \texttt{gourd init} do the following:

\begin{enumerate}
  \item Add a new folder under \texttt{src/resources/gourd\_init\_examples}.
  \item Put all of the example files (including a valid \texttt{gourd.toml}) in that folder.
  \item The table of all examples is contained in \texttt{src/gourd/init/builtin\_examples.rs}
  in the function called \texttt{get\_examples()}, add your example there with the name
  of the tar file being the same as the examples name.
\end{enumerate}

Here is an example entry from the table:

\begin{verbatim}
"a-simple-experiment",
InitExample {
    name: "A Simple Experiment",
    description: "A comparative evaluation of two simple programs.",

    directory_tarball: include_bytes!(concat!(
        env!("OUT_DIR"),
        "/../../../tarballs/a_simple_experiment.tar.gz"
    )),
},
\end{verbatim}
