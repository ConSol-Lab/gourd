\pagebreak


\section{Architecture}\label{sec:architecture}

\subsection{Technology}

This section lists the technologies used in the software and its development.

\subsubsection{Rust}

Rust is the main programming language used for \gourd\ and its components.
The project uses Rust\footnote{See the MSRV for the version of Rust required}
edition 2021 for buidling.

\subsubsection{SLURM}

A key part of \gourd\ is interaction with the \emph{Slurm Workload Manager}.
Slurm is open-source software that runs on most supercomputers (including
DelftBlue). The application interfaces with Slurm's stable command-line
interface and generates job descriptions using its batch scripting language.

\subsection{Structural Overview}

This section explains each part of the application's modular layout.
The Rust source code for the application is in the \verb|./src/| directory.

The \gourd\ project is divided into three core crates:
\gourd, \gourdlib, and \gourdwrap.
Their individual responsibilities are laid out in this section.

\subsubsection{\gourd\ -- Command-Line Application}

The \gourd\ command-line application is the core part of the project.
The source code is located in \verb|./src/gourd/|.

The \gourd\ binary uses the \gourdlib\ shared library to interface with
the wrapper, \gourdwrap\.
The responsibilities of the \gourd\ binary are to interact with the user,
schedule and run experiments (at a high level), and collect aggregate
status and metrics from the experiment's runs.

\subsubsection{\gourdwrap\ -- Wrapper Program}

The \gourdwrap\ is a binary that should not be invoked manually by the user.
The source code is located in \verb|./src/gourd-wrapper/|.
It is responsible for the low-level implementation of executing a run;
it takes care of combining runs of an individual binary program and input,
encapsulating the program in platform-native frameworks for collecting metrics.
This means that \gourdwrap\ is the actual executable scheduled once for each run.

\subsubsection{\gourdlib\ -- Wrapper Interface}

The \gourdlib\ crate contains all data shared between the application and the wrapper.
The source code is located in \verb|./src/gourd-lib/|.
This includes the `gourd.toml' configuration file (the formal definition of an experiment)
and an `experiment.lock' runtime data file.
The general pattern is that \gourd\ writes a `<experiment-number>.lock' TOML file when
an experiment is started, and \gourdwrap\ subsequently reads only the `lock' file and
on each execution to determine the path of the executable, the resource limits,et cetera


\subsection{Modules of the \gourd\ command-line application}

This section covers the most important modules of the \gourd\ command-line application
(the crate with source code in \verb|./src/gourd/|).

\subsubsection{CLI}
The command-line interface of \gourd\ is the main entry-point of the
application.
The user uses sub-commands and interactive prompts to control
the software.
Internally, this is done using the Rust `clap' Command-Line
Argument Parser module.
The interactions are defined in
\verb|./src/gourd/cli/| where \verb|mod.rs| is the module root.

\subsubsection{SLURM}

The `slurm' module is where interactions with SLURM are defined.
The \verb|mod.rs| file defines the \verb|SlurmInteractor| trait;
this is used to abstract away from the specific mode of interacting
with Slurm.
Currently, the only implementation (`SlurmCli') is located in the
\verb|interactor.rs| file and targets the SLURM command-line interface.
We decided to use the SLURM CLI rather than the C libraries or
the REST API because of its stability and simplicity.
However, the \verb|SlurmInteractor| trait abstracts away from this and
adding an implementation that interacts with SLURM through another
channel simply requires conforming to the trait.
