\section{Building and distributing}

\textbf{All} of {\gourd}s builds are orchestrated exclusively by \texttt{cargo}
and all of the rules for non standard builds are contained in the \texttt{build.rs} file.


\subsection{Features}

\gourd contains two optional compilation features described below:

\begin{itemize}
  \item \texttt{documentation}
  This when turned on will compile all of the documentation for \gourd\ and
  the resulting files will be placed in \texttt{target/release/manpages}.

  \item \texttt{fetching}
  This feature when turned on will compile the https backend for fetching resources
  in experiments, this was made an optional feature because it requires linking against
  OpenSSL and increases compilation time considerably.

  This is tuned on by default, to turn it on specify: \texttt{--no-default-features}
\end{itemize}

\subsection{Building}

To build \gourd\ in release mode issue:

\texttt{\$ cargo build --release}
\vspace{0.5cm}

\noindent The following actions will be taken:

\begin{itemize}
  \item The shell completions\footnote{For shells: \texttt{fish}, \texttt{zsh}, \texttt{bash}, and PowerShell. See \texttt{build.rs}}
  will be compiled and placed in \texttt{target/release/completions}
  \item \gourdlib\ will be compiled.
  \item \gourd\ will be compiled and placed in \texttt{target/release/gourd}.
  \item \gourdwrap\ will be compiled and placed in \texttt{target/release/gourd\_wrapper}.
\end{itemize}

You can now run \gourd\ form the \texttt{target/} folder.

\subsection{Documentation}
\label{sec:docbuild}

To build \gourd\ with documentation issue:

\texttt{\$ cargo build --release --features documentation -vv}
\vspace{0.5cm}

\noindent This requires Xe\LaTeX, \texttt{latex2man}, as well as \texttt{mandoc(1)} installed.

This will build all of the manpages to: \texttt{PDF}, \texttt{HTML}, and \texttt{groff}.
This will also build this(!) document.

All of the resulting files are placed in: \texttt{target/release/manpages}.


\subsection{Distribution}
\textcolor{red!30!black}{\textbf{
  It was decided that at the current time \gourd\ will not be compiled into
  operating system specific packages. This is left for future versions of the application.
}}
\vspace{0.3cm}

For users to be able to easily install and use a fully setup gourd a specially prepared script was made.

Rust builds by default leak information about the system they were built on.
As such we recommend, when building for distribution to other users, to use the command:

\texttt{\$ RUSTFLAGS="--remap-path-prefix \$HOME=/REDACTED/" cargo build --release --features documentation -vv}

This will redact all paths containing the username, more paths to redact can be added space-separated.

This build will also produce: \\
\texttt{target/release/generate-installer.sh}.

By moving into the \texttt{target/release/} folder and invoking the bourne shell script like so:

\begin{verbatim}
$ cd target/release/
$ ./generate-installer.sh
\end{verbatim}

Two platform specific files will be generated \texttt{install.sh}, and \texttt{uninstall.sh}.

These files are fully bundled versions of gourd and can be distributed directly to users.
There are some enviormental variables that need to be set when generating the script,
please continue to \ref{sec:installexample1} and \ref{sec:installexample2}.

\subsubsection{An example}
\label{sec:installexample1}

Let us look at the generation process for the platform \texttt{x86\_64-unknown-linux-gnu}\footnote{aka Linux systems with the GNU C library on amd64 platforms}.

\begin{verbatim}
$ INSTALL_PATH="/usr/local/bin" \
  MANINSTALL_PATH="/usr/local/share/man" \
  FISHINSTALL_PATH="/usr/share/fish/completions" \
  ./generate-installer.sh
\end{verbatim}

During generation of the installation script one has to specify the paths for placing the binaries, manpages, and shell completions.
These will differ between systems or distibutions that is why this was introduced.

After running this we will get two scripts: \texttt{install-x86\_64-unknown-linux-musl.sh}, \texttt{uninstall-x86\_64-unknown-linux-musl.sh}.

Now the scrips can be distributed and the results of running them are:

\begin{verbatim}
$ sudo ./install-x86_64-unknown-linux-gnu.sh
sudo (user@pc) password:
gourd --> /usr/local/bin/gourd
gourd_wrapper --> /usr/local/bin/gourd_wrapper
manpages/gourd.1.man --> /usr/local/share/man/man1/gourd.1
manpages/gourd.toml.5.man --> /usr/local/share/man/man5/gourd.toml.5
manpages/gourd-tutorial.7.man --> /usr/local/share/man/man7/gourd-tutorial.7
Installing completions... this can fail and thats fine!
completions/gourd.fish --> /usr/share/fish/completions/gourd.fish
\end{verbatim}

\begin{verbatim}
$ sudo ./uninstall-x86_64-unknown-linux-gnu.sh
removed '/usr/local/bin/gourd'
removed '/usr/local/bin/gourd_wrapper'
removed '/usr/local/share/man/man1/gourd.1'
removed '/usr/local/share/man/man5/gourd.toml.5'
removed '/usr/local/share/man/man7/gourd-tutorial.7'
Uninstalling completions... this can fail and thats fine!
removed '/usr/share/fish/completions/gourd.fish'
\end{verbatim}


\subsubsection{Notes on DelftBlue}
\label{sec:installexample2}

Of course these folders for placing binary files and others will differ from system to system
that is why the option of specifying them is provided, a user can also override these paths by
passing the same variables (\texttt{INSTALL\_PATH, MANINSTALL\_PATH, } $\dots$) to the install script.

So for example on DelftBlue, where users are not allow to install applications
to the global \texttt{bin} one can do:

\begin{verbatim}
$ INSTALL_PATH="~/.local/bin" \
  ./install-x86_64-unknown-linux-gnu.sh
\end{verbatim}

And the binaries will be placed accordingly.
