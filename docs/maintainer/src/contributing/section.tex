\pagebreak


\section{Contributing}

The contribution guidelines for this project, and the decisions made about code style,
will be outlined in this section of the document.

\subsection{Version Control}

The project is hosted on GitLab.
Changes to the ``main'' branch must be staged through a merge request approved by at least two maintainers.
The number of commits should be kept to a minimum to ensure clarity.
All commit messages must be lowercase.

Currently, the ``main'' branch is not public and can contain new features that do not increment the software version.
However, each merge request that increments the software version must do so in ``Cargo.toml'', version strings in the project,
the manpage, and the maintainer documentation at \texttt{version-history/section.tex} and \texttt{preamble.tex}.

\subsection{Documentation}

In addition to adding an entry to the version history, as detailed above, maintainers should also make sure that the
documentation remains thorough.
\begin{enumerate}
  \item The Maintainer PDF documentation (this document) contains an overview of the application's underlying design.
  \item The ``manpage'' (in \verb|docs/user/gourd.man|) contains detailed usage instructions and configuration documentation.
  \item Inline documentation (Rustdoc) contains additional information necessary for the developer of a particular feature.
\end{enumerate}

\subsection{Code Style}

Code written in Rust should be formatted using `cargo fmt` and validated using `cargo clippy`.
Beware when using \verb|[cfg]| statements, as they can cause `clippy' issues on different platforms.
All relevant code style rules are enforced by the GitLab CI and can be tested locally
by configuring the Git hooks:
\begin{lstlisting}
git config --local core.hooksPath .hooks/
\end{lstlisting}

\subsection{Testing}

The project goal is to be thoroughly tested with full-coverage systematic unit tests.
The GitLab environment analyzes coverage. This is displayed alongside each merge request.
